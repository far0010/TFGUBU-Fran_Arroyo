\apendice{Competencias de Sostenibilidad Curricular}

Como ya se expuso brevemente en el capítulo 3.8 de la memoria, durante el desarrollo de este trabajo y la aplicación \textbf{GeNomIn}, que ha sido creada para modernizar un proceso administrativo tedioso. Desarrollada en \acrshort{APEX} y desplegada en \acrshort{OCI}, se ha intentado integrar los \acrfull{ODS} en el ámbito tecnológico ~\cite{MarkiegiIntegrandoODSGrado}.

Este proyecto me ha permitido comprender y aplicar estas competencias en sostenibilidad curricular, que se alinean con el \textbf{uso responsable de recursos}, la\textbf{ conciencia ambiental}, la \textbf{participación comunitaria} y los \textbf{principios éticos}.

\section{Competencias Adquiridas}
Este \acrshort{TFG}, ha permitido reflexionar sobre la transformación que está sufriendo la sociedad, en particular las administraciones, influidas por la tecnología. En particular vemos como la digitalización reduce el impacto ambiental, mejora el acceso a la información y es mucho más eficiente que el mero uso del papel.

\subsection{Uso sostenible de recursos}

Según los datos aportados por (\href{https://www.oracle.com/es/sustainability/}{Oracle y sostenibilidad}) sus centros utilizan energía renovable y diseño circular. Además, la eliminación de procesos manuales y en papel, supone una reducción de un 90\% de éstos, haciendo el proceso más sostenible en un entorno en la nube.

\subsubsection{Participación comunitaria}
En el desarrollo del proyecto, como no podía ser de otra forma, se involucró al personal del Servicio para un mejor control del proceso administrativo seguido en el proyecto. Esto promueve una transformación digital transparente dentro de las instituciones.

\subsubsection{Pricipios éticos}
El uso, por parte de Oracle, de herramientas \acrfull{ESG}, refuerza el compromiso ético con la sostenibilidad, alienado con los abjetivos \acrshort{ODS}, \textbf{12: (producción
y consumo responsables)}, \textbf{13: Acción por
el clima} y \textbf{16: Paz, justicia e instituciones sólidas}

\section{Aplicación en el Proyecto GeNomIn}
\subsection{Diseño y Funcionalidad}
\textbf{GeNomIn} ha sido diseñada para sustituir hojas de cálculo en almacenamiento compartido por una solución segura, escalable e integral. Esta automatización puede reducir los errores hasta en un 70\% y un ahorro temporal de más del 30\%
\subsection{Impacto Ambiental y Social}
La digitalización de procesos administrativos es uno de los objetivos \acrshort{ODS} (9), que complementa al objetivo (13), \textbf{acción climática}, y a las \textbf{instituciones sólidas} (16). Así, esta migración a la nube supone, con servicios optimizados con gran eficiencia energética, un impacto ambiental mínimo.
\subsection{Conciencia ambiental}
Durante la realización del proyecto se reflexionó sobre la ideoneidad de la migración a un entorno en la nube como Oracle y el consumo de sus servidores. Pero, evidentemente, el impacto ambiental es mucho mayor en el entorno de la propia Universidad que no dispone de los recursos de Oracle, aunque tampoco se ha podido valorar esta diferencia.

\section{Conclusión}

Lo cierto es que antes de la iniciación de este \acrshort{TFG}, no se había reflexionado sobre el impacto que puede tener la tecnología en el medio ambiente. Todos oímos hablar del coste (economico-ambiental) de los grandes servidores de Google, pero no caemos en la cuenta de nuestro propio trabajo.
Analizando el uso anterior, a través de hojas de cálculo almacenamientos, impresiones en papel y procesos obsoletos, nos damos cuenta de la importancia de adquirir estas competencias, no solo en el ámbito de la informática, si no, en la vida cotidiana.

\textbf{GeNomIn} es solo una gota de agua en la transformación de la administración hacia procesos más modernos y eficientes, pero me ha servido de aprendizaje para comprender el impacto que tienen las decisiones técnicas que se alinean con los \acrshort{ODS} y la agenda 2030.