\capitulo{6}{Trabajos relacionados}
Para la realización de este trabajo se hicieron diversas consultas a través de buscadores, de aplicaciones similares, que hicieran esta función de aunar los proyectos de investigación, sus convocatorias y sus contratos emitiendo informes personalizados.
Así, existen en el mercado aplicaciones como \href{https://www.pandadoc.com/}{PandaDoc} o \href{https://www.zoho.com/es-xl/contracts/?lb=es-xl}{Zoho Contracts} que gestionan contratos, pagos, etc..., pero no están vinculados a convocatorias de proyectos de investigación.

En este sentido y ya conocido,  \href{https://www.universitasxxi.com/}{Universitas XXI}, sí ofrece varios módulos dedicados a la gestión de RRHH, proyectos de Investigación y gestión económica, con emisión de varios informes independientes, pero que no se adaptan por completo a los requerimientos de los usuarios.
Veamos en esta tabla comparativa entre la aplicación \textbf{GeNomIn} y los módulos de \textbf{\acrshort{UXXI}}:

\begin{table}[h!]
	\centering
	\footnotesize
	\begin{tabular}{|l|l|l|l|l|}
		\hline
		\rowcolor{gray!20}
		\multicolumn{1}{c}{\textbf{Funcionalidad}\rule{0pt}{25pt}} & \multicolumn{1}{c}{\textbf{GeNomIn}} & \multicolumn{1}{c}{\textbf{UXXI-Inv}} &
		\multicolumn{1}{c}{\textbf{UXXI-Eco}} &
		\multicolumn{1}{c}{\textbf{UXXI-RRHH}} 
		\\
		\hline
		Contratos Inv. & Parcial & Sí & No & No\\
		Convocatorias Inv. & Parcial & Sí & No & No\\
		Ges. Investigadores & Parcial & Sí & No & No\\
		Informes nómina & Sí & Sí & General & Contable\\
		Contratos con pagos & Sí & NO & General & Contable\\
		Informes Person. & Sí & Propio & Propio & Propio\\
		Infraestructura & Oracle Cloud & Propio & Propio & Propio\\
		Flexibilidad & Alta & Versiones & Versiones & Versiones\\
		Costo & Free Tier & Licencia & Licencia & Licencia\\
		\hline
	\end{tabular}
	\caption{Comparativa GeNomIn vs Universitas XXI}
	\label{tab:comGenUniv}
\end{table}

Como podemos comprobar la opción comercial tiene muchísimo más desarrollo pero no se personaliza a los objetivos de cada usuario, teniendo que esperar a la liberación de versiones, si es que está incluida esa actualización o solicitar una personalización, que generalmente no es barata.