\capitulo{3}{Conceptos teóricos}
\section{Introducción}
En este capítulo se van a presentar los fundamentos teóricos en los que se basa esta aplicación, cómo nace la necesidad y surge la idea de su desarrollo.

\section{Antecedentes} 
La idea de la realización de este \acrshort{TFG}, surge principalmente de la relación laboral de este estudiante con la Universidad, concretamente en el Servicio de Gestión de la Investigación. Como ya se comentó en la introducción de esta memoria, en este Servicio se realizan diferentes tareas relacionadas
con la investigación y particularmente contratos asociados a proyectos de investigación.

Para la realización de este tipo de contratos los Investigadores, realizan una solicitud de contratación, luego solicitantes con diversas características
requeridas (sobre todo titulación), participan en las mismas y para el que es seleccionado, se realiza un contrato asociado a un proyecto, \textbf{que tiene que
estar en vigor}. 

Hasta aquí, cabría esperar que finalizase este proceso, pero hay que tener en cuenta las disponibilidades presupuestarias de los proyectos y los cambios que pueden surgir durante la vida del contrato.

\section{Problemática}
Una vez generado el contrato, el Servicio de Recursos Humanos genera mensualmente dos nóminas para cada contratado, una de seguridad social y otra de la nómina en sí. Estamos hablando que actualmente hay más de cuatrocientos contratos vinculados a estos proyectos.

Estás nóminas (\textbf{sábanas}), son remitidas al Servicio de Investigación para que de conformidad a la existencia de crédito, la cuantía y validez del proyecto, antes de ordenar efectuar el pago general a contabilidad.\label{img: sabana}
\imagen{sabana}{Sábana de nómina de RRHH}{1
} 

Para ésto, la Sección de Personal Contratado del Servicio de Investigación al realizar un contrato, a punta en una hoja Excel los datos del contratado, el proyecto vinculado, las mensualidades de un año con su seguridad social y observaciones.\label{img: contrato}
\imagen{tablaexcel}{Tabla de nómina}{1}

Cuando recibe las sábanas de nómina, coteja cada referencia del proyecto (\textbf{\textbf{orgánica}}), con los datos que tiene apuntados en su hoja de cálculo y comprueba si hay diferencias, si los datos del proyecto son correctos así como las cuantías totales por orgánica.

Esto proceso puede repetirse varias veces si hay errores. Al finalizar el año se hace una copia de la tabla de Excel del año anterior y se copia en el siguiente, donde se añaden los nuevos contratos.

\section{Proyecto GeNomin}

En este estado de cosas y con la oportunidad de realizar un \acrshort{TFG} de utilidad, se propone transformar la hoja de cálculo en una \acrshort{BDR} en Oracle, que unifique todos los datos que ahora se dispersan en hojas de cálculo anuales y a través de una aplicación en \acrshort{APEX}, permita obtener informes de nómina que faciliten el trabajo de cotejo con las sábanas de retribuciones. Con la particularidad de que la aplicación de Gestión de Proyectos, Convocatorias y Contratos, se desarrolla también sobre ambas plataformas, no supondría un problema de adaptación para los usuarios.

En el proyecto, se podrán gestionar investigadores, proyectos, convocatorias y solicitantes. Asociar un solicitante a una convocatoria y que generar las nóminas correspondientes. Prorrogar y renunciar a los contratos (modificando la nóminas a pagar) Y finalmente generar diversos informes de utilidad.