\apendice{Diseño GeNomIn}
\section{Introducción}
En este Anexo C voy a explicar como se ha realizado la aplicación \textbf{GeNomIn}, a través de la plataforma \acrfull{APEX} 2402.
Se detallará como se han implementado las diferentes páginas para obtener las funcionalidades y requisitos exigidos por los usuarios.

Como ya se indicó en el \textbf{apartado 4.2} de la memoria, una vez instalado \acrshort{APEX} y la base de datos \acrfull{ODB}, es preciso crear los usuarios que utilizarán la aplicación, en este caso, a parte del \textbf{Administrador}, se crearon un \textbf{desarrollador} y un \textbf{usuario} final.
\imagenDos{userapex}{Usuarios APEX}{1}

\section{App Builder}
La herramienta para construcción de aplicaciones de apex ~\cite{OracleAPEXAdministration}, es un \acrshort{IDE} con poco código, a través de páginas y componentes compartidos (\textbf{shared components}).
Inicialmente podemos crear nuestra apliación a través de varias fuentes; con ficheros (XML, CSV XLSX, JSON, SQL), copiando aplicaciones existentes o directamente con el asistente, que nos creará una página de lógin y nos irá permitiendo añadir páginas y componentes para personalizar nuestro diseño.
\imagen{appbuilder}{Creación de app APEX}

En los siguientes apartados se irán detallando los aspectos fundamentales para el desarrollo de las páginas necesarias e importantes en este proyecto y el código asociado, ya sea en \acrshort{SQL}, \acrshort{PL/SQL} o \acrshort{JS}.\\ 
Todo este código también está disponible en el repositorio \href{https://github.com/far0010/TFGUBU-Fran_Arroyo/tree/main/project-docs/memoria/sql}{Código de GeNomIn}

\section{Página Proyectos}
En esta página se detallan todos los proyectos que existen en la base de datos a través de una consulta simple \acrshort{SQL}. El gestor de \textbf{acciones} de \acrshort{APEX}, permite ejecutar búsquedas y diversos filtros, además de añadir, eliminar registros y actualizar datos.\\ 
Se puede ver la consulta detallada en \href{https://github.com/far0010/TFGUBU-Fran_Arroyo/blob/main/project-docs/memoria/sql/consulta_proyectos.sql}{consulta proyectos}
\section{Página Responsables}
Al igual que la página anterior usa una consulta \acrshort{SQL}, para mostrar todos los responsables, tengan o no, proyectos abiertos. También permite realizar \textbf{acciones}. 

\section{Página Mantenimiento Convocatorias}
En esta página se hace una consulta \acrshort{SQL} que devuelve el detalle de todas las convocatorias y su responsable asociado, incluyendo \textbf{acciones}.
\subsection{Crear Convocatoria}
Para crear una nueva convocatoria ha sido preciso el uso de primero, una consulta \acrshort{SQL}, que nos muestra todos los investigadores y segundo, una consulta que una vez escogido el investigador, muestre los proyectos abiertos para ese investigador.\\
Luego se precisa de dos acciones dinámicas en \acrshort{JS}, en las que se controla, primero, si el investigador tiene proyectos abiertos desde la fecha actual, sino, muestra mensaje de aviso, y la segunda, nos muestra el tiempo restante del proyecto para tener en cuenta para la duración de contratos.\\ 
Para ver los scripts \href{https://github.com/far0010/TFGUBU-Fran_Arroyo/blob/main/project-docs/memoria/sql/scrips_crear_convo.js}{creación de convocatoria}

\imagenDos{actconvo}{Actions para crear convocatoria}{.8}

\section{Página de Solicitantes}
Al igual que las anteriores, muestra una búsqueda \acrshort{SQL} simple, con todos los solicitantes y si tienen o no una convocatoria asignada. 
\subsection{Crear solicitante}
Para la creación de un nuevos solicitantes, hay que tener en cuenta varios aspectos, como son:\\
Si quedan plazas disponibles en la convocatoria, esto se ha resuelto con una búsqueda \acrshort{SQL} que cuenta los solicitantes de las convocatorias y las muestra si son mayores de 0.\\
Luego es preciso verificar si la convocatoria está abierta y que envíe un mensaje de aviso en caso de estar cerrada. Esto se ha resuelto con una primera \textbf{acción dinámica} y con consulta \acrshort{SQL} que guarda el campo abierto en un \textbf{textbox} oculto. Para ello, se utilizan las opciones de la página \acrshort{APEX}: \textbf{ítems to submit} y \textbf{Affected Elements}, que nos permiten enviar y guardar los datos.
aje de error.\\
El último paso es para comprobar si el solicitante tiene la titulación que 	requiere la convocatoria, procediendo de forma similar, con otra \textbf{acción dinámica}. Primero una consulta sencilla \acrshort{SQL} que guarda la titulación requerida en un campo oculto, \textbf{\textbf{P7\_TIT-REQ}} y luego mostramos un aviso con \acrshort{JS}, en el caso de que la titulación del solicitante no sea igual o superior a la requerida en la convocatoria.\\
Se puede consultar el codigo en \href{https://github.com/far0010/TFGUBU-Fran_Arroyo/project-docs/memoria/sql
	/consultas_solicitante.sql}{código crear solicitante}
\imagenDos{comtitrequerida}{Action que comprueba titulación}{.9}

\section{Página de Contratos}
Este primer \textbf{informe interactivo} no requiere de consultas adicionales ya que \acrshort{APEX}, realiza esta función son solo indicarle la tabla que queremos mostrar, en este caso CONTRATOS.
\subsection{Página Nuevo Contrato}
Para la realización de un nuevo contrato tendremos que realizar diversas comprobaciones.\\ Inicialmente ofrecemos las convocatorias disponibles, \acrshort{APEX} también nos ofrece esta opción sin necesidad de programación adicional a través de \textbf{List of Values: CONVOCATORIA.TITULO}, en  (shared componentes).
\imagenDos{listofvalues}{List of Values}{0.8}
En una primera consulta \acrshort{SQL}, obtendremos los solicitantes que participan en la convocatoria elegida anteriormente.
Ahora necesitamos tres \textbf{acciones dinámicas}, una que va a cargar las fechas de inicio y fin del proyecto, en campos ocultos, P12\_F\_INI\_OCULTO y P12\_F\_FIN\_OCULTO  a través de dos consultas \acrshort{SQL} y una otra en \acrshort{JS}, para forzar de nuevo la comprobación en caso de error.

Una vez cargadas las fechas de inicio y fin de los proyectos es preciso cotejarlas con las introducidas por el usuario, para comprobar si están dentro del margen del proyecto a través de dos nuevas  \textbf{acciones dinámicas} en \acrshort{JS}, la primera controlará que el inicio del contrato sea posterior al inicio del proyecto, y la segunda comprueba si la fecha de fin de contrato, es posterior a la de inicio y si sobrepasa al fin del proyecto.

Realizadas las comprobaciones, es preciso el paso más importante, generar las nóminas para el contrato en la tabla nómina.\\

Para ello se requiere un \textbf{proceso} en \acrshort{PL/SQL}. Este comprobará las fechas de inicio y fin e irá añadiendo nóminas y seguridades sociales al contrato, más una seguridad social más, ya que se pagan a mes vencido.\\
El código detallado en \href{https://github.com/far0010/TFGUBU-Fran_Arroyo/project-docs/memoria/sql
	/consultas_contratos.sql}{código de generación de contratos}
\imagenDos{insercionnominas}{Proceso inserción de nóminas}{1}
\subsection{Página Renovación}
La renovación de un contrato requiere de varios controles. 

Inicialmente tendremos que seleccionar los contratos con fecha fin posterior a la actual, a través de una consulta \acrshort{SQL}, en el campo P13\_SEL\_CONT.

Seguidamente a través de una \textbf{acción dinámica}, mostraremos los campos de ese contrato seleccionado anteriormente, con otra consulta \acrshort{SQL}.

Guardaremos, dentro de esta misma acción dinámica, la fecha fin en un campo oculto P13\_NEW\_FECHA, para su posterior comparación. 

Como en anteriores ocasiones esto es posible por la propiedad de \acrshort{APEX},\textbf{ Affected elements}, que permite guardar el resultado de una acción o consulta en un determinado campo.

Una vez obtenidos los datos del contrato en vigor, se introduce la fecha de la renovación, que tiene que comprobarse a través de otra \textbf{acción dinámica} asociada a un botón \textbf{verifica-fecha}, que cambiará a color verde si es correcta y mostrará error en caso contrario. 

Esta acción dinámica es verificada por dos códigos \acrshort{JS}.
Este primer código es bastante más complejo, ya que ha requerido la conversión de fechas. Debe comprobar que la fecha de renovación es posterior a la actual y anterior al fin del proyecto. Si todo es correcto cambiará el color del botón a verde.
Como en anteriores casos, si la fecha fuera errónea, habría que reactivar el campo de fecha.

Finalmente, como en el caso inicial de creación de nóminas, necesitamos ejecutar un \textbf{procedimiento} en \acrshort{PL/SQL} que genere las nuevas nóminas adicionales, que irán desde la nueva fecha fin anterior hasta la nueva introducida.
Para ver el código asociado \href{https://github.com/far0010/TFGUBU-Fran_Arroyo/project-docs/memoria/sql
	/consultas_renovacion.sql}{código renovaciones}

\subsection{Página Renuncia al contrato}
Para realizar la renuncia de un contrato, primero se seleccionan aquellos contratos que tienen una fecha posterior a la actual con una \textbf{acción dinámica}, tal y como en el caso anterior de la \textbf{renovación} a través de una consulta \acrshort{SQL} y guardando el valor de la seg. social en un campo oculto P14\_SS\_OCULTA, ya que es un valor que necesitaremos después, mediante la segunda parte de esta acción en \acrshort{JS}.

Esta primera parte nos presentará los datos del contrato a renunciar, debiendo introducir después, la fecha de renuncia, que se verificará a través de otra acción dinámica asociada al botón, \textbf{Verificar Fecha}, que comprueba si la fecha de renuncia introducida es menor que la actual fecha fin y si es posterior a la inicial, a través de código \acrshort{JS}.

Una vez verificada la fecha, con el botón en verde, y pulsado \textbf{Renunciar}, se ejecuta un \textbf{proceso} \acrshort{PL/SQL} que reduce las nóminas hasta la nueva fecha fin y también cambia la de fin del contrato. \\ El código se puede consultar en \href{https://github.com/far0010/TFGUBU-Fran_Arroyo/project-docs/memoria/sql
	/consultas_renuncia.sql}{código renuncias}

\section{Página de Informes}
En las siguientes páginas se muestran los informes realizados.
\subsection{Página Nómina-mes}
Este informe es el objetivo final de la aplicación, para ello se solicitan primeramente un mes y un año y se realiza una consulta \acrshort{SQL} al pulsar el botón \textbf{Consultar} que devuelve los datos de nóminas para ese mes/año y la suma total del mes para todas las aplicaciones mostradas.

De este informe se guarda el total en P18\_T\_GEN, a través de una \textbf{acción dinámica} en \acrshort{JS}. Esto lo podría hacer el propio informe dinámico, pero nos interesa para poder imprimirlo en el informe pdf.

Mostrado el informe, podemos filtrar la orgánica que queramos y que vuelva a calcular el importe total de éstas. Primero mostramos todas las referencias que salen en el informe del mes, con una consulta \acrshort{SQL} y luego con la acción dinámica anterior se vuelve a refrescar el importe total.

Para poder imprimir el informe a PDF, a través de \acrfull{AOP}, es precisa otra acción dinámica más, que dispare el \textbf{plug-in} \textit{UC-APEX OfficePrint(AOP)-DA}, al pulsar el botón \textbf{Generar PDF}.

Al plug-in, hay que indicarle cuál es nuestra plantilla personalizada, guardada en shared components  /Application Files: \textbf{plantilla\_informe\_nomina.docx}

Este  código  se puede consultar en  \href{https://github.com/far0010/TFGUBU-Fran_Arroyo/blob/main/project-docs/memoria/sql/consultas_inf_nom_mes.sql}{código de generación informes-mes}

\imagenDos{generapdf}{Action generar PDF}{1}

\subsection{Página Vencimientos}
En este informe interactivo se presentarán los vencimientos de contratos entre dos fechas para su control, a través de una consulta \acrshort{SQL}, que al pulsar el botón \textbf{Consulta} activará la acción dinámica de \textbf{submit} al informe.
Ver consulta en \href{https://github.com/far0010/TFGUBU-Fran_Arroyo/project-docs/memoria/sql
	/consultas_inf_vencimientos.sql}{código de consultas vencimientos}

\subsection{Página Contratos}
Esta página es un informe interactivo simple que muestra los datos de todos los contratos a través de una consulta \acrshort{SQL}, pudiendo realizar los filtros predeterminados que ofrece \textbf{Actions}.
Ver la consulta en \href{https://github.com/far0010/TFGUBU-Fran_Arroyo/blob/main/project-docs/memoria/sql/consulta_informe_contratos.sql}{código informe de contratos}

\subsection{Página Nóminas-Periodo}
Aquí se presentan para un contratado las nóminas que se han pagado en un periodo de tiempo solicitado.\\

Para ello se realiza una primera consulta \acrshort{SQL} que muestra los contratados en una lista desplegable.
Una vez seleccionado el contratado e introducidas las fechas de inicio y fin se realiza otra consulta \acrshort{SQL} para rellenar el informe, mostrando el total pagado en ese periodo. En este caso se ha suprimido la opción \textbf{Actions}, ya que el informe es único.\\

El código completo en  \href{https://github.com/far0010/TFGUBU-Fran_Arroyo/blob/main/project-docs/memoria/sql/consultas_nom_periodo.sql}{código nóminas por periodo}