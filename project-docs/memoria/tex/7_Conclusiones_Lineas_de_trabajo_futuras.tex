\capitulo{7}{Conclusiones, valoración personal y trabajo futuro}
En este capítulo se detalla si se han cumplido los Objetivos del proyecto, tanto funcionales, no funcionales y personales, y si se han desarrollado y consolidado las competencias adquiridas durante el Grado.

\section{Cumplimiento de objetivos funcionales}
El principal objetivo de este \acrshort{TFG}, era la transformación de una hoja de cálculo Excel, en la que se contabilizan las nóminas del personal contratado (ver imagen \ref{img: contrato}), en una \acrshort{ODB} y la generación de una aplicación en \acrshort{APEX}, que además de llevar la gestión de las convocatorias y contratos, presentara un informe para el cotejo con la retribución emitida por RRHH (ver imagen \ref{img: sabana}).

Una vez finalizado el desarrollo, el proyecto ha logrado cumplir estos objetivos, desarrollando una aplicación \textbf{Genomin}, que cumple a la perfeción todos los retos planteados.
\imagen{logo}{Logotipo de la aplicación: GeNomIn}{.2}

\section{Cumplimiento de objetivos no funcionales}
Dentro de los objetivos no funcionales se pretendía reforzar las competencias adquiridas durante el Grado, así como explorar nuevas herramientas de desarrollo y ampliar algunas de las conocidas. Todos éstos retos han sidos superados desde la gestión de proyectos con la \gls{Metodología Ágil}, generación de \acrshort{ODB}, \acrshort{SSL}, creación de la aplicación en \acrshort{APEX} 24, realización de test de \acrshort{VYP},  hasta su despliegue en \textbf{Oracle Cloud} y todo ello documentado y apoyado con el repositorio documental \textbf{GitHub}

\section{Valoración Personal}
Antes de empezar este \acrshort{TFG}, tenía la sensación de que era un mero trámite más para la consecución del Grado en Informática, pero lo cierto es que es una de las asignaturas más enriquecedoras. Ya no solo supone un reto en cuanto al desarrollo de código, que hoy en día no lo es tanto con la irrupción de la \acrshort{IA}, si no que supone un esfuerzo organizativo, que requiere de muchas de las competencias adquiridas durante la carrera. Es aquí donde se comprenden muchos conceptos, como la \gls{Normalización}, \gls{MER}, la gestión de proyectos, el uso de \gls{Metodología Ágil}, la \acrfull{VYP} etc., en el entorno de desarrollo de un proyecto.
En mi caso la realización de este proyecto una vez finalizado el curso y completadas el resto de asignaturas, me ha permitido dedicar mucho más tiempo, organización e investigación que de otra manera hubiera resultado muchísimo más complejo.

Con todo esto, la aplicación web \textbf{GeNomIn} desarrollada, desde mi punto de vista, cumple con los objetivos inicialmente planteados y podría ser desplegada perfectamente para su uso en el Servicio para el que ha sido diseñada, puesto que corre sobre \acrfull{ODB}, la BD utilizada por la Universidad y teniendo como plataforma \acrshort{APEX}, utilizado por \acrshort{UXXI}-Investigación (actualmente proveedor de la Universidad de Burgos), lo cual no supondría ningún problema de adaptabilidad por parte de los usuarios.

\section{Trabajo futuro}
El trabajo abordado en este \acrshort{TFG}, nace de la necesidad observada en la gestión y cotejo de pagos al personal investigador contratado, con cargo a proyectos de investigación.
En este sentido se ha abordado la generación de este informe (\textbf{Pago nómina-mes}), otros como la renovación y la renuncia de contratos, e informes adicionales que muestran información sobre vencimientos de contratos en un periodo determinado y de carácter general.

Quedaría por abordar, las \textbf{subidas salariales} y \textbf{bajas médicas} que también tienen influencia en los pagos mensuales, que tal y como está ahora mismo la aplicación, se podrían realizar modificando directamente los pagos en la tabla correspondiente (\textbf{NOMINAS}), pero sin tener una gestión directa.
Seguramente, si se llegase a poner en explotación, sería imprescindible añadir otro tipo de \textbf{informes personalizados}, que surgieran del propio uso de la aplicación y la gestión diaria.

Evidentemente este proyecto simplemente es para uso de forma local, por el  propio personal del Servicio de Investigación que gestiona contratos, ya que la plataforma \acrshort{UXXI} desarrolla sus propios módulos bajo licencia, pero puesto que se ha considerado su necesidad podría ser planteado para desarrollos futuros a la propia empresa.

\section{Reflexión sobre el uso de \acrshort{IA}}
Durante la realización de este trabajo se han realizado diversas consultas a través de \gls{Copilot} sobre todo en "\textbf{momentos de pánico}". En este sentido cabe destacar, que en las versiones libres (la utilizada en este proyecto), son un mero apoyo y consulta, ya que normalmente las respuestas obtenidas ofrecen más una orientación, que una respuesta efectiva y que generalmente pueden llevar al error, sobre todo cuando se realizan, por ejemplo, consultas complejas.
Evidentemente no se pueden poner puertas al campo y el uso de la \acrshort{IA} está cada vez más extendido, sería deseable su inclusión en las asignaturas del Grado, como apoyo al desarrollo y al conocimiento, como en su día fueron, los libros, Internet y ahora esta nueva tecnología. Tendremos que adaptarnos todos.


