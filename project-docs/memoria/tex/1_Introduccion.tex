\capitulo{1}{Introducción}

El Servicio de Gestión de la Investigación realiza diversas tareas en el ámbito universitario que van desde la tramitación de facturas, ayudas a la investigación, doctorado, gestión de proyectos, becas y contratos de investigación, pudiendo considerarse en sí misma una \textbf{mini-gestora}   universitaria.

En este contexto cobran una gran trascendencia los contratos de investigación, ya que generan un gran volumen de trabajo e información. Así, para a realización de un contrato de investigación asociado a un proyecto, entran en liza diferentes ámbitos del servicio que pueden llegar ser inmanejables tal y como se realiza actualmente.

Se tramitan aproximadamente 400 contratos de investigación, contabilizando su detalle a través de una hoja de cálculo, que contiene en sus campos los principales datos, junto con las retribuciones mensuales, tales como: orgánica, datos, contratado y cada una de las mensualidades del año, con su seguridad social.

Antes de finalizar el mes, se cotejan los pagos apuntados en esta tabla excel, con los datos remitidos por el Servicio de Retribuciones, lo cual, como es de suponer, genera no pocos errores.

Se pretende con este trabajo de fin de Grado, ya no sólo facilitar el que hacer diario, si no evitar errores que se producen fácilmente con el manejo tan voluminoso de hojas de cálculo.

Dicha migración se realizará a través de una aplicación en \acrfull{APEX} 24.2 (sobre Oracle 23ie) con la que está familiarizada el personal de este servicio, ya que actualmente se ha migrado a esta plataforma, tanto el portal de investigación con la gestión económica.
