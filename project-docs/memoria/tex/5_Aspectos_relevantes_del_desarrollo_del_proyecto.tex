\capitulo{5}{Aspectos relevantes del desarrollo del proyecto}
En este capítulo se detallan los aspectos que han influido en el desarrollo de la aplicación \textbf{GeNomIn}, tanto técnicas, económicas y de oportunidad.

\section{Oportunidad de negocio}
Cuando se aborda un \acrshort{TFG}, según mi opinión, debe poder trasladarse al ámbito profesional y así desarrollar las competencias adquiridas durante estos años de aprendizaje en el Grado.
En el entorno laboral en el que desarrollo mi trabajo, existía la necesidad de un control más férreo y menos tedioso de los pagos de nómina a través de proyectos de Investigación que se estaban realizando a través de hojas de cálculo. En este sentido, cabe explicar que no es que se pague mal, si no en el proceso de comprobación utilizado, en  lo relativo a los contratos realizados con proyectos de investigación.
Mensualmente se emite un listado de los pagos que corresponden a estos proyectos y se cotejan contra una hoja de Excel, y con más de 400 contratos, es evidente que resulta inviable. (ver imagen \ref{img: contrato})
Así pues, se planteó la propuesta de crear una aplicación, que no sólo ofreciera un tipo de listado más uniforme, si no que permitiera un control más adecuado de las personas contratadas, las convocatorias y sus responsables.
En este sentido, hay que indicar, que la plataforma que actualmente utiliza la Universidad, ofrece la gestión de convocatorias y contratos, pero no es posible el control de pagos de estos contratos con los módulos instalados.

\section{¿Por qué APEX 2402?}
Uno de los aspectos más importantes y que influyen en el desarrollo de una aplicación es la herramienta que se utiliza. En este sentido y puesto que el entorno de trabajo se ha desarrollado con \acrshort{APEX}, parecía bastante adecuado, tratando así de que la transición de la hoja de cálculo a la aplicación sea lo menos costosa para los usuarios.

Como vimos en la sección dedicada a esta herramienta \ref{sec: apex},  \acrshort{APEX} ofrece muchas características técnicas que ya de por sí solas la hacen elegible, pero veamos qué facilita para el desarrollo y la gestión.
\subsection{Integración con \acrfull{ODB}}
\acrshort{APEX} aprovecha la potencia y eficiencia de \acrshort{ODB} así como sus características avanzadas así como integración nativa con \acrshort{SQL} y \acrshort{PL/SQL} desarrollado por Oracle ~\cite{OracleAPEXSQL}
\subsubsection{Gestión de Usuarios}
\imagen{apex_acl}{Gestión de usuarios de APEX}{1}
Como vemos en la imagen anterior \acrshort{APEX}, permite un control total de los usuarios a través de:
\begin{itemize}
	\item \textbf{\acrfull{ACL}}: permiten asignar roles asuarios específicos.
	\item \textbf{Esquemas de autorización}: que se aplican a páginas o componentes individuales para restringir acceso.
	\item \textbf{Generación automática de componentes}: al activar este \acrshort{ACL}, se generan automáticamente distintos componentes; administración de usuarios, región de control de acceso, roles, compilaciones condicionales, etc. ~\cite{OracleAPEXApp}, \cite{OracleAPEXAdministration}
\end{itemize}
\section{Creación de las Páginas}
\acrshort{APEX} combina el desarrollo visual con la lógica declarativa y programación libre, para el desarrollo de aplicaciones web a través del \textbf{diseñador de páginas} (entorno visual para definir la estructura y componentes) y del \textbf{asistente de creación} (para generar formularios, informes interactivos, dashboards, etc.)
\imagen{apex_pag}{Vista de las páginas APEX}{1}
\section{Programación}
\acrshort{APEX} permite la programación de procesos y acciones dinámicas en \acrshort{PL/SQL}, \acrshort{JS} sin complejidad de código.
Además se pueden establecer condiciones y validaciones para mostrar u ocultar elementos según reglas establecidas.
\imagen{apex_act}{Acciones dinámicas APEX y su código js}{1}
En la imagen superior podemos ver un grupo de acciones dinámicas asociadas a la pag 12 del proyecto: \textbf{Nuevo contratado}, y se muestra el código de la que verifica si la fecha de inicio del contrato, es posterior a la fecha de inicio del proyecto asociado.
\section{Despliegue en Oracle Cloud Free Tier}
Una de las características principales por las que se eligió este modelo de desarrollo en el binomio \acrshort{APEX} \ \acrshort{ODB}, fue la gratuidad y facilidad de los servicios. Aunque como ya se explicó en el capítulo 4 (ver \ref{sec: OFT}), el alta en Cloud, no es sencillo, se prefirió a otras opciones como \textbf{kubernetes}, con una curva de aprendizaje y configuración mayores o \textbf{virtualbox} que hubiera valido para un entorno local, exportando la imagen y confiando en la configuración del usuario de destino.

Así, para este tipo de desarrollos más \textit{experimentales}´ se recomienda su uso, ya que además permite interactuar con entorno profesional, de bajo coste, alta disponibilidad, escalable (previo pago) y seguridad integrada, todo ello ofrecido por Oracle Cloud. Ver imagen(\ref{img: OCloud})


