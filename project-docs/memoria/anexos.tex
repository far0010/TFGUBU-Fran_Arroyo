\documentclass[a4paper,12pt,twoside]{memoir}

% Castellano
\usepackage[spanish,es-tabla]{babel}
\selectlanguage{spanish}
\usepackage[utf8]{inputenc} 
\usepackage[T1]{fontenc}
\usepackage{lmodern} % scalable font
\usepackage{microtype}
\usepackage{placeins}
\usepackage{float}
\usepackage{needspace}
\usepackage{listings}
\usepackage{textcomp}
\RequirePackage{booktabs}
\RequirePackage[table]{xcolor}
\RequirePackage{xtab}
\RequirePackage{multirow}

\RequirePackage{todonotes}

% Añado listings para ver el código en la memoria mejor
\usepackage{listings}
\usepackage[utf8]{inputenc}
\usepackage{array}
\renewcommand{\arraystretch}{1.1} 

\lstdefinelanguage{JavaScript}{
	keywords={const, let, var, function},
	keywordstyle=\color{blue}\bfseries,
	comment=[l]{//},
	morecomment=[s]{/*}{*/},
	commentstyle=\color{gray}\ttfamily,
	stringstyle=\color{orange}\ttfamily,
	morestring=[b]",
	morestring=[b]',
	literate={ñ}{{\~n}}1
}


\lstdefinelanguage{PLSQL}{
	keywords={
		BEGIN, END, EXCEPTION, CURSOR, PROCEDURE, FUNCTION, RETURN, IS, LOOP, WHILE, IF, THEN, ELSE, ELSIF,
		DECLARE, OPEN, FETCH, CLOSE, EXIT, GOTO, NULL, RAISE, WHEN, CASE, SELECT, INSERT, UPDATE, DELETE,
		INTO, FROM, WHERE, AND, OR, NOT, COMMIT, ROLLBACK
	},
	keywordstyle=\color{purple}\bfseries,
	ndkeywords={
		NUMBER, VARCHAR2, DATE, BOOLEAN, INTEGER, SYS_REFCURSOR, OUT, IN, AS
	},
	ndkeywordstyle=\color{blue}\bfseries,
	identifierstyle=\color{black},
	sensitive=true,
	comment=[l]{--},
	morecomment=[s]{/*}{*/},
	commentstyle=\color{gray}\ttfamily,
	stringstyle=\color{orange}\ttfamily,
	morestring=[b]',
	morestring=[b]"
}

\lstdefinelanguage{SQL}{
	basicstyle=\ttfamily\small,
	keywordstyle=\color{blue}\bfseries,
	commentstyle=\color{gray},
	stringstyle=\color{orange},
	literate={€}{{\euro}}1
}


% Links
\PassOptionsToPackage{hyphens}{url}\usepackage[colorlinks]{hyperref}
\hypersetup{
	allcolors = {red}
}

% Acrónimos
\usepackage[acronym]{glossaries}
\makenoidxglossaries
\newacronym{APEX}{APEX}{\textit{Application Express}}
\newacronym{TFG}{TFG}{\textit{Trabajo de Fin de Grado}}
\newacronym{ODB}{ODB}{\textit{ORACLE 23ai}}
\newacronym{FN}{FN}{\textit{Forma Normal}}
\newacronym{SGBD}{SGBD}{\textit{Sistemas de Gestión de BD}}
\newacronym{HTTPS}{HTTPS}{\textit{Secure Hipertext Transfer Protocol}}
\newacronym{PBD}{PBD}{Plugable Data Base}
\newacronym{VYP}{VYP}{\textit{Validación y Pruebas}}
\newacronym{BDR}{BDR}{\textit{Base de Datos Relacional}}
\newacronym{PK}{PK}{\textit{Primary Key-Clave Primaria}}
\newacronym{FK}{FK}{\textit{Foreing Key-Clave Externa}}
\newacronym{IDE}{IDE}{\textit{Entorno de Desarrollo Integrado}}
\newacronym{DDL}{DDL}{\textit{Lenguaje de Definición de Datos}}
\newacronym{DML}{DML}{\textit{Lenguaje de Manipulación de Datos}}
\newacronym{SQL}{SQL}{\textit{Structured Query Language}}
\newacronym{MPV}{MPV}{\textit{Mínimo Producto Viable}}
\newacronym{ODS}{ODS}{\textit{Objetivos de Desarrollo Sostenible}}
\newacronym{ORDS}{ORDS}{Oracle REST Data Services}
\newacronym{REST}{REST}{REpresentational State Transfer}
\newacronym{SSL}{SSL}{Secure Sockets Layer}
\newacronym{OCI}{OCI}{Oracle Cloud Infrastructure}
\newacronym{VM}{VM}{Máquina Virtual}
\newacronym{AOP}{AOP}{APEX Office Print}
\newacronym{PL/SQL}{PL/SQL}{Procedural Language/Structured Query Language}
\newacronym{IA}{IA}{Inteligencia Artificial}
\newacronym{ACL}{ACL}{Listas de Control de Acceso}
\newacronym{JS}{JS}{JavaScript}
\newacronym{UXXI}{UXXI}{Universitas XXI}
\newacronym{RGPD}{RGPD}{Reglamento General de Protección de Datos}
\newacronym{ESG}{ESG}{Ambientales, Sociales y de Gobernanza}
\newacronym{MVC}{MVC}{Modelo Vista Controlador}
\newacronym{SOA}{SOA}{Arquitectura Orientada a Servicios}

\newglossaryentry{Persistencia desacoplada}{
	name=Persistencia desacoplada,
	description={Modelo donde el almacenamiento se gestiona de forma independiente a la lógica de negocio, permitiendo flexibilidad en la infraestructura}
}

\newglossaryentry{Milestone}{
	name=Milestone,
	description={Punto de control o evento significativo dentro de un proyecto que marca la finalización de una fase importante o la consecución de un objetivo específico}
}

\newglossaryentry{RESTFUL}{
	name=Representational State Transfer,
	description={Estilo de arquitectura para diseñar servicios web que se comunican a través de HTTP}
}

\newglossaryentry{Copilot}{
	name=Copilot,
	description={Asistente de inteligencia artificial desarrollado por Microsoft, diseñado para ayudar a los usuarios en diversas tareas a través de diferentes aplicaciones}
}

\newglossaryentry{Datos MOCK}{
	name=Datos MOCK,
	description={Datos ficticios para probar aplicaciones, bases de datos, interfaces, sin necesidad de usar datos reales o confidenciales}
}

\newglossaryentry{LATEX}
{
	name=LATEX,
	description={Sistema de composición de textos orientado a la creación de documentos escritos, genralmente científicos que presenten una alta calidad tipográfica}
}

\newglossaryentry{Cardinalidad}
{
	name=Cardinalidad,
	description={Relación entre el número de instancias de una entidad con respecto a otra entidad con la que se relaciona}
}

\newglossaryentry{CasodeUso}
{
	name=Caso de Uso,
	description={Descripción de cómo los actores(usuarios), interactúan con el sistema para lograr un objetivo específico}
}

\newglossaryentry{Atomicidad}
{
	name=Atomicidad,
	description={Propiedad de un campo, por el cual no puede ser descompuesto en pedazos más pequeños por el DBMS}
}

\newglossaryentry{Normalización}
{
	name=Normalización,
	description={Proceso de estructuración los datos en una BD relacional, eliminando redundancia de datos y manteniendo la integridad a través de Formas Normales}
}

\newglossaryentry{MER}
{
	name=Modelo Relacional,
	description={Herramienta que permite organizar los datos en tablas y las relaciones entre las mismas}
}

\newglossaryentry{SLL}
{
	name=Secure Sockets Layer,
	description={protocolo de seguridad que cifra la comunicación entre un navegador web y un servidor, protegiendo la información transmitida}
}

\newglossaryentry{Release}
{
	name= Release,
	description={Versión específica de un software que ha sido completada, probada y puesta a disposición para su uso por los usuarios finales.}
}

\newglossaryentry{Kanban}
{
	name= Kanban,
	description={Método de gestión visual del flujo de trabajo que se caracteriza por el uso de tableros visuales y tarjetas para representar las tareas y su progreso a través de diferentes etapas, promoviendo la transparencia y la mejora continua.}
}

\newglossaryentry{Metodología Ágil}
{
	name= Metodología ágil,
	description={Enfoque iterativo e incremental para la gestión de proyectos, especialmente popular en el desarrollo de software, que se centra en la flexibilidad, la colaboración y la entrega continua de valor. En lugar de adherirse rígidamente a un plan predefinido, agile abraza el cambio y la adaptación a medida que avanza el proyecto.}
}

\newglossaryentry{Scrum}
{
	name= Scrum,
	description={Marco de trabajo ágil diseñado para gestionar y completar proyectos de manera eficiente y colaborativa.}
}

\newglossaryentry{Sprint}
{
	name= Sprint,
	description={Período de tiempo corto y fijo (generalmente de una a cuatro semanas) durante el cual un equipo trabaja para completar un conjunto específico de tareas.}
}

\newglossaryentry{Licencia MIT}
{
	name= Licencia MIT,
	description={Licencia de software libre y permisiva, desarrollada por el MIT, que permite a cualquiera usar, copiar, modificar, distribuir y vender el software con muy pocas restricciones, siempre y cuando se incluya la nota de copyright y la copia de la licencia en las distribuciones}
}

\newcommand{\autor}{Francisco José Arroyo Redondo}

% Ecuaciones
\usepackage{amsmath}

% Rutas de fichero / paquete
\newcommand{\ruta}[1]{{\sffamily #1}}

% Párrafos
\nonzeroparskip

% Huérfanas y viudas
\widowpenalty100000
\clubpenalty100000

% Evitar solapes en el header
\nouppercaseheads

% Imagenes
\usepackage{graphicx}

\newcommand{\imagen}[2]{
	\begin{figure}[!!htbp]
		\centering
		\includegraphics[width=0.9\textwidth]{#1}
		\caption{#2}\label{fig:#1}
	\end{figure}
	\FloatBarrier
	
}

\newcommand{\imagenDos}[3]{
	\begin{figure}[!h]
		\centering
		\includegraphics[width=#3\textwidth]{#1}
		\caption{#2}\label{fig:#1}
	\end{figure}
	\FloatBarrier
}

\newcommand{\imagenflotante}[2]{
	\begin{figure}%[!t]
		\centering
		\includegraphics[width=0.9\textwidth]{#1}
		\caption{#2}\label{fig:#1}
	\end{figure}
}

\newcommand{\imagenminipage}[2]{
	\centering
	\includegraphics[width=0.9\textwidth]{#1}
	\textbf{Figura:} #2
}



% El comando \figura nos permite insertar figuras comodamente, y utilizando
% siempre el mismo formato. Los parametros son:
% 1 -> Porcentaje del ancho de página que ocupará la figura (de 0 a 1)
% 2 --> Fichero de la imagen
% 3 --> Texto a pie de imagen
% 4 --> Etiqueta (label) para referencias
% 5 --> Opciones que queramos pasarle al \includegraphics
% 6 --> Opciones de posicionamiento a pasarle a \begin{figure}
\newcommand{\figuraConPosicion}[6]{%
  \setlength{\anchoFloat}{#1\textwidth}%
  \addtolength{\anchoFloat}{-4\fboxsep}%
  \setlength{\anchoFigura}{\anchoFloat}%
  \begin{figure}[#6]
    \begin{center}%
      \Ovalbox{%
        \begin{minipage}{\anchoFloat}%
          \begin{center}%
            \includegraphics[width=\anchoFigura,#5]{#2}%
            \caption{#3}%
            \label{#4}%
          \end{center}%
        \end{minipage}
      }%
    \end{center}%
  \end{figure}%
}

%
% Comando para incluir imágenes en formato apaisado (sin marco).
\newcommand{\figuraApaisadaSinMarco}[5]{%
  \begin{figure}%
    \begin{center}%
    \includegraphics[angle=90,height=#1\textheight,#5]{#2}%
    \caption{#3}%
    \label{#4}%
    \end{center}%
  \end{figure}%
}
% Para las tablas
\newcommand{\otoprule}{\midrule [\heavyrulewidth]}
%
% Nuevo comando para tablas pequeñas (menos de una página).
\newcommand{\tablaSmall}[5]{%
 \begin{table}
  \begin{center}
   \rowcolors {2}{gray!35}{}
   \begin{tabular}{#2}
    \toprule
    #4
    \otoprule
    #5
    \bottomrule
   \end{tabular}
   \caption{#1}
   \label{tabla:#3}
  \end{center}
 \end{table}
}

%
%Para el float H de tablaSmallSinColores
%\usepackage{float}

%
% Nuevo comando para tablas pequeñas (menos de una página).
\newcommand{\tablaSmallSinColores}[5]{%
 \begin{table}[H]
  \begin{center}
   \begin{tabular}{#2}
    \toprule
    #4
    \otoprule
    #5
    \bottomrule
   \end{tabular}
   \caption{#1}
   \label{tabla:#3}
  \end{center}
 \end{table}
}

\newcommand{\tablaApaisadaSmall}[5]{%
\begin{landscape}
  \begin{table}
   \begin{center}
    \rowcolors {2}{gray!35}{}
    \begin{tabular}{#2}
     \toprule
     #4
     \otoprule
     #5
     \bottomrule
    \end{tabular}
    \caption{#1}
    \label{tabla:#3}
   \end{center}
  \end{table}
\end{landscape}
}

%
% Nuevo comando para tablas grandes con cabecera y filas alternas coloreadas en gris.
\newcommand{\tabla}[6]{%
  \begin{center}
    \tablefirsthead{
      \toprule
      #5
      \otoprule
    }
    \tablehead{
      \multicolumn{#3}{l}{\small\sl continúa desde la página anterior}\\
      \toprule
      #5
      \otoprule
    }
    \tabletail{
      \hline
      \multicolumn{#3}{r}{\small\sl continúa en la página siguiente}\\
    }
    \tablelasttail{
      \hline
    }
    \bottomcaption{#1}
    \rowcolors {2}{gray!35}{}
    \begin{xtabular}{#2}
      #6
      \bottomrule
    \end{xtabular}
    \label{tabla:#4}
  \end{center}
}

%
% Nuevo comando para tablas grandes con cabecera.
\newcommand{\tablaSinColores}[6]{%
  \begin{center}
    \tablefirsthead{
      \toprule
      #5
      \otoprule
    }
    \tablehead{
      \multicolumn{#3}{l}{\small\sl continúa desde la página anterior}\\
      \toprule
      #5
      \otoprule
    }
    \tabletail{
      \hline
      \multicolumn{#3}{r}{\small\sl continua en la pagina siguiente}\\
    }
    \tablelasttail{
      \hline
    }
    \bottomcaption{#1}
    \begin{xtabular}{#2}
      #6
      \bottomrule
    \end{xtabular}
    \label{tabla:#4}
  \end{center}
}

%
% Nuevo comando para tablas grandes sin cabecera.
\newcommand{\tablaSinCabecera}[5]{%
  \begin{center}
    \tablefirsthead{
      \toprule
    }
    \tablehead{
      \multicolumn{#3}{l}{\small\sl continúa desde la página anterior}\\
      \hline
    }
    \tabletail{
      \hline
      \multicolumn{#3}{r}{\small\sl continúa en la página siguiente}\\
    }
    \tablelasttail{
      \hline
    }
    \bottomcaption{#1}
  \begin{xtabular}{#2}
    #5
   \bottomrule
  \end{xtabular}
  \label{tabla:#4}
  \end{center}
}



\definecolor{cgoLight}{HTML}{EEEEEE}
\definecolor{cgoExtralight}{HTML}{FFFFFF}

%
% Nuevo comando para tablas grandes sin cabecera.
\newcommand{\tablaSinCabeceraConBandas}[5]{%
  \begin{center}
    \tablefirsthead{
      \toprule
    }
    \tablehead{
      \multicolumn{#3}{l}{\small\sl continúa desde la página anterior}\\
      \hline
    }
    \tabletail{
      \hline
      \multicolumn{#3}{r}{\small\sl continúa en la página siguiente}\\
    }
    \tablelasttail{
      \hline
    }
    \bottomcaption{#1}
    \rowcolors[]{1}{cgoExtralight}{cgoLight}

  \begin{xtabular}{#2}
    #5
   \bottomrule
  \end{xtabular}
  \label{tabla:#4}
  \end{center}
}




\graphicspath{ {./img/} }

% Capítulos
\chapterstyle{bianchi}
\newcommand{\capitulo}[2]{
	\setcounter{chapter}{#1}
	\setcounter{section}{0}
	\setcounter{figure}{0}
	\setcounter{table}{0}
	\chapter*{#2}
	\addcontentsline{toc}{chapter}{#2}
	\markboth{#2}{#2}
}

% Apéndices
\renewcommand{\appendixname}{Apéndice}
\renewcommand*\cftappendixname{\appendixname}

\newcommand{\apendice}[1]{
	%\renewcommand{\thechapter}{A}
	\chapter{#1}
}

\renewcommand*\cftappendixname{\appendixname\ }
\renewcommand\lstlistlistingname{Índice de Códigos}
\renewcommand\lstlistingname{Código}
% Formato de portada
\makeatletter
\usepackage{xcolor}
\newcommand{\tutor}[1]{\def\@tutor{#1}}
\newcommand{\course}[1]{\def\@course{#1}}
\definecolor{cpardoBox}{HTML}{E6E6FF}
\def\maketitle{
  \null
  \thispagestyle{empty}
  % Cabecera ----------------
\noindent\includegraphics[width=\textwidth]{cabecera}\vspace{1cm}%
  \vfill
  % Título proyecto y escudo informática ----------------
  \colorbox{cpardoBox}{%
    \begin{minipage}{.8\textwidth}
      \vspace{.5cm}\Large
      \begin{center}
      \textbf{TFG del Grado en Ingeniería Informática}\vspace{.6cm}\\
      \textbf{\LARGE\@title{}}
      \end{center}
      \vspace{.2cm}
    \end{minipage}

  }%
  \hfill\begin{minipage}{.20\textwidth}
    \includegraphics[width=\textwidth]{escudoInfor}
  \end{minipage}
  \vfill
  % Datos de alumno, curso y tutores ------------------
  \begin{center}%
 	{%
 		\noindent\LARGE
 		Presentado por \@author{}\\ 
 		en Universidad de Burgos --- \@date{}\\
 		Tutor: \@tutor{}\\
 	}%
 \end{center}%
 \null
 \cleardoublepage
 }
 \makeatother
 
 \newcommand{\nombre}{Francisco J. Arroyo Redondo}
 
 
 % Datos de portada
 \title{\fontsize{18pt}{22pt}\selectfont Gestión de Contratos GeNomIn\\
 \fontsize{16pt}{18pt}\selectfont Aplicación APEX sobre Oracle, para la gestión de contratos del Personal contratado con cargo a proyectos de Investigación}
 \author{\nombre}
 \tutor{Pedro Renedo Fernández}
 \date{Septiembre 2025}
\setcounter{tocdepth}{4}
\begin{document}

\maketitle



\cleardoublepage


%%%%%%%%%%%%%%%%%%%%%%%%%%%%%%%%%%%%%%%%%%%%%%%%%%%%%%%%%%%%%%%%%%%%%%%%%%%%%%%%%%%%%%%%



\frontmatter


\clearpage

% Indices
\tableofcontents

\clearpage

\listoffigures

\clearpage

\listoftables

\clearpage

\lstlistoflistings
\clearpage

\mainmatter

\appendix

\apendice{Plan de Proyecto Software}

\section{Introducción}

La realización de la planificación de este proyecto software ha sido gestionada a través de la \textit{\gls{Metodología Ágil}} dividiendo el trabajo en hitos, \textbf{\gls{Milestone}s}, y éstos a su vez en \textbf{\gls{Sprint}s}.

Esta división del proyecto en tareas, permite desarrollo flexible y adaptativo, produciendo entregas parciales del producto por cada finalización de \gls{Sprint}.

En particular en este proyecto se ha utilizado una mezcla de metodologías; \textbf{\gls{Scrum}} ~\cite{SchwaberGuiaDefinitivaScrum2020}, que prioriza el desarrollo en \gls{Sprint}s, con sus roles y eventos definidos y \textbf{\gls{Kanban}}, que se centra en la visualización de los flujos de trabajo en un tablero con columnas, etiquetadas con las diferentes fases del desarrollo, en las cuales se van colocando tarjetas con las tareas que se están realizando y que se han asociado a los \gls{Sprint}s.

Para una mejor organización temporal, se han establecido una serie de hitos (\gls{Milestone}s), y éstos a su vez en \textbf{\gls{Sprint}s}, que agrupan un conjunto de \gls{Sprint}s reflejando las diferentes etapas del proyecto, como comprobamos en la planificación temporal siguiente:

\section{Plan temporal}
Como se ha indicado anteriormente para una mejor gestión de tiempos se ha dividido el proyecto en diferentes hitos, \gls{Sprint}s y todo ello gestionado a través del \gls{Kanban}:
\subsection{Kanban}
El tablero \gls{Kanban} ha sido una de las herramientas fundamentales para el desarrollo de esta aplicación. Gestionado por la plataforma \textbf{Zube} e integrado con \textbf{GitHub}, permite un control de las diferentes tareas. Para ello se dividen en varios estados, en el caso de este proyecto; inbox (entrada por defecto), \textbf{backlog} (lista inicial de trabajos a realizar por el equipo), \textbf{ready} (marca el inicio del \gls{Sprint} para esa tarjeta), \textbf{in progress} (durante el desarrollo), \textbf{in review} (revisión del sprint) y finalmente \textbf{done}, cuando se cierra ese \gls{Sprint}.
Vemos un ejemplo del \gls{Kanban} del proyecto.
\imagen{kanban1}{Detalle del tablero kanban al inicio del proyecto}
\subsection{\gls{Milestone}}
\begin{itemize}
	\item \textbf{Kick-off: Puesta en marcha del proyecto}:
	Completado el 16 de junio. Una vez reunido con el tutor y definidas las líneas generales del proyecto se procede a recopilar e instalar las herramientas necesarias para el desarrollo y la documentación.
	Este hito hubo que extender hasta el 23, por el upgrade a \textbf{\acrshort{APEX}2402}
	\item \textbf{Prototipo}: Completado el 30 de junio. Aunque se han ido realizando diversas `entregas', se establece un prototipo que sirva de base para que el `usuario' verifique si se está yendo en la dirección correcta en el desarrollo. Simplemente se muestran las primeras funcionalidades y los diversos menús que se podrán usar. Queda reflejado como el \textbf{1er \gls{Release}}
	\item \textbf{\acrshort{MPV}:Mínimo producto viable}: Completado el 14 de julio. Este \textbf{2º \gls{Release}} ya presentan prácticamente todas las funcionalidades de la aplicación, pudiendo observar los usuarios, si es ajustado a sus requerimientos.
	\item \textbf{Desarrollo completo}: Completado el 15 de agosto. Se ha finaliza la aplicación (\textbf{3er \gls{Release}}) y despliega en Oracle cloud, siendo funcional a todos los efectos.
	\item \textbf{Documentación}: Se prepara, completa y revisa toda la documentación del proyecto; memoria, anexos, vídeos y el resto de documentos para presentar.
\end{itemize}
\subsection{Sprints}
Para la ejecución del proyecto se han realizado los siguientes \gls{Sprint}s, algunos coincidentes en el tiempo dentro del mismo hito:
\begin{itemize}
	\item \textbf{Kick-off: Puesta en marcha del proyecto hasta 16 de junio}:
	\subitem \textbf{Análisis}: en este \gls{Sprint} se realizaron las tareas de análisis de datos de la \textbf{hoja de cálculo excel} y su \textbf{\gls{Normalización}}, \textbf{definición las entidades} y sus \textbf{relaciones} para finalmente crear las distintas \textbf{tablas} con sus campos, definiendo \textbf{\acrshort{PK}} y \textbf{\acrshort{FK}} 
	\subitem{\textbf{Instalación de herramientas}}: Se abrió el repositorio en \textbf{GitHub} ~\cite{ChaconProGitTodo} para el proyecto, instalándose en modo local, \textbf{\acrshort{APEX} 5.01} (posteriormente v.23ai) ~\cite{DattaInstallingOracleDatabasea} y \textbf{Oracle 9} (posteriormente 24.2) ~\cite{JenningsInstallingConfiguringAPEX} , herramientas que se habían utilizado en la asignatura de \textit{Sistemas de Gestión de Bases de Datos}. También se instaló \textbf{TestCafé} ~\cite{TestCodeGuide} para la ejecución de test, con el complemento (Allure) así como el editor de LaTex, \textbf{TexStudio} ~\cite{BibliotecaComplutenseLaTeXTuTFG2024}.
    \imagenDos{kickoff}{Milestone: Kick-Off}{1}
	\item{\textbf{Prototipo: hasta el 30/06}}:
	\subitem{\textbf{Creación pantalla de inicio}}: Se crea la pantalla de \textbf{log-in} y la página de entrada principal, así como el primer módulo de Administración, que contendrá los datos de los \textbf{proyectos} de los \textbf{responsables}. Como los datos obrantes en el sistema son privados, es preciso para la realización de las pruebas la creación de \textbf{\gls{Datos MOCK}}.
	\subitem{\textbf{Módulos y test}}: En este \gls{Sprint} se diseña un programa en Phyton para rellenar las tablas de responsables y proyectos con datos de prueba y rellenar las tablas correspondientes para poder realizar los test. Se diseñan los primeros Test para los menús creados que son la base del prototipo. Se realiza la migración de \textbf{HTTP} a \textbf{\acrshort{HTTPS}}, debiéndose obtener un certificado a través de \textbf{OpenSSL}
	\imagen{prototipo}{Milestone: Prototipo}
	\item{\textbf{MPV: hasta el 14/07}}: 
	\subitem{\textbf{Creación Contrato}}: Se crean la utilidad principal para efectuar un contrato a un solicitante. El proceso debe crear las nóminas correspondientes en la tabla \textbf{NOMINAS}.
	\subitem{\textbf{Creación de renovación y renuncias}}: Se crean los módulos de renovación, que amplía la fecha de fin de contrato y añade las nóminas correspondientes y el de renuncia, que recorta la fecha de contrato y elimina la nóminas desde esa fecha hasta el final del contrato anterior.
	\subitem{\textbf{Test de usabilidad}}: Se testean estas nuevas funcionalidades, comprobando que se actualizan correctamente las fechas y las nóminas.
	\imagen{mpv}{Mileston: MPV}
	\item{\textbf{Desarrollo Completo: hasta 15 de agosto}}:
	\subitem{\textbf{Creación de informes}}: Se crean los informes pendientes, \textbf{Nómina-mes}, \textbf{Vencimientos} y \textbf{Nóminas periodo}.
	\subitem{\textbf{Creación de cuenta Oracle Cloud}}: Se crea la cuenta de \acrshort{OCI} y las instancias para la BD y \acrshort{APEX}. Se efectúa el traspaso de la base de datos local a Cloud y se instala la aplicación MPV, para su prueba.
	\subitem{\textbf{Revisión de Test e informes}}: Se crea e instala \acrfull{AOP} ~\cite{Oracle-maxParte122021}. Se realizan los test de éstos últimos informes, para comprobar que se generan correctamente y se emite el informe en PDF a través de \acrshort{AOP}
	\subitem{\textbf{Creación de \acrfull{OCI}}}: Se finaliza el desarrollo de la aplicación y se despliega en \acrshort{OCI}. En este caso no hace falta certificado \acrshort{SSL} ni para Cloud ni para \acrshort{AOP}, ya que al ser el mismo proveedor confía en su entorno.
	\imagen{kcloud}{Milestone: Desarrollo Completo}
	\item{Documentación: hasta el 5 de septiembre}: 
	\subitem{\textbf{Generación de documentación}}: Se finaliza la memoria del proyecto, documento creciente que se va desarrollando a lo largo del proyecto, pero es en este punto final cuando se invierte la mayoría del tiempo en su finalización, plasmándose los \acrfull{ODS} ~\cite{MarkiegiIntegrandoODSGrado}. Para la realización de la memoria y los anexos, ha sido imprescindible el aprendizaje de \gls{LATEX}. Finalizado el proyecto se crea el manual de uso y se sube al \textbf{wiki} de GitHub.
	Para completar la documentación se crean los vídeos de funcionamiento de la app \textbf{GeNomIn} y de presentación del \acrshort{TFG}
	\imagen{kdocument}{Milestone: Documentación}
	
	\item{\textbf{Gráfica de Velocidad}}: en la siguiente gráfica podemos ver cómo se han ido realizando los diferentes \gls{Sprint}s, manteniendo una constancia relativa. Se puede observar, que la tarea de \textbf{verficación y despliegue} ha sido más laboriosa, ya que comprendía la finalización y trasvase de datos a Cloud.
	\imagen{velocidad}{Gráfica de Velocidad de los Sprints}
\end{itemize}
\clearpage
Vemos también en la imagen, como las tareas han sido sincronizadas con el repositorio de \textbf{GitHub}, incluyendo su \gls{Sprint} y \gls{Milestone} asignados.

\imagenflotante{tarea2}{Tarea \#2:Instalación de herramientas en GitHub}

\subsection{Revisión del sprint}
Tras cada \textbf{sprint} el código es analizado por \textbf{SonarCube}, para detectar errores, vulnerabilidades, problemas de estilo y garantizando la calidad. Así mismo, cada funcionalidad añadida por un sprint ha sido validada por \textbf{TestCafe} cuyos resultados se presentan visualmente en los informes generados por \textbf{Allure} (informe completo \href{https://far0010.github.io/TFGUBU-Fran_Arroyo/informe/#} {aquí})
\imagenDos{allure}{Informe de test Allure}{.85}


\clearpage
\section{Estudio de Viabilidad}

\subsection{Viabilidad Técnica}
Este proyecto ha sido realizado siguiendo la filosofía de software libre y coste cero, lo que garantiza su viabilidad técnica, no obstante se incluyen los costes reales de implantación. Hay que tener en cuenta que la Universidad de Burgos tiene licencia con Oracle y servidores propios.

\begin{table}[ht]
	\centering
	\begin{tabular}{|p{3cm}|p{4cm}|p{4cm}|}
		\hline
		\rowcolor{gray!20}
		\textbf{Herramienta} & \textbf{Función} & \textbf{Licencia / Coste real} \\
		\hline
		Oracle 23ai & BD relacional & Desde 43.700 € proc. + 22\% mant. anual \\
		\hline
		SQL Developer & Diseño BD & Gratuito \\
		\hline
		Oracle APEX 24.02 & Desarrollo web & Free en local/Desde 112.24 €/mes en OCI \\
		\hline
		\acrshort{AOP} & Informes PDF personalizados & 100 inf. gratis, luego desde 35 €/mes \\
		\hline
		TestCafe & Pruebas & Gratuito \\
		\hline
		Oracle Cloud Free Tier & Despliegue Cloud & Gratuito (limitado) / Desde 110.4 €/mes \\
		\hline
		TeXstudio & Redacción técnica & Gratuito \\
		\hline
	\end{tabular}
	\caption{Coste real estimado de herramientas utilizadas en el proyecto GeNomIn}
\end{table}
\subsection{Viabilidad Económica}
Para el desarrollo del proyecto, se estiman unas 300 horas a un precio estimado de 25 €/hora para un técnico cualificado, más las cuotas de autónomo. Además se incluye el alquiler de local y los gastos asociados de luz. Para el equipamiento se ha calculado sobre una amortización de 4 años del equipamiento y la licencia de  \acrshort{AOP} en producción. 
El coste de la \gls{Licencia MIT} de la aplicación es gratuita.
\begin{table}[htbp]
	\centering
	\begin{tabular}{|p{5cm}|>{\centering\arraybackslash}p{5cm}|}
		\hline
		\rowcolor{gray!20}
		\textbf{Concepto} & \textbf{Importe €}\\
		\hline
		Horas de desarrollo (300 h x 25 €/h) & 7.500,00 € \\
		\hline
		Cuota autónomos (4 meses x 230 €/mes) & 920,00 € \\
		\hline
		Alquiler espacio de trabajo (4 meses x 250 €/mes) & 1.000,00 € \\
		\hline
		Electricidad (4 meses x 20 €/mes) & 80,00 € \\
		\hline
		Amortización portátil (4 meses) & 100,00 € \\
		\hline
		Licencia de la app (MIT) & 0,00 € \\
		\hline
		AOP – Apex Office Print (4 meses x 35 €/mes) & 140,00 € \\
		\hline
		\textbf{Total} & \textbf{9.740,00 €} \\
		\hline
	\end{tabular}
	\caption{Coste económico actualizado del proyecto GeNomIn}
\end{table}

\subsection{Viabilidad Legal}
\begin{itemize}
	\item \textbf{Propiedad intelectual}: Este desarrollo ha sido realizado por el autor del \acrshort{TFG}, sin restricciones.
	\item \textbf{Protección de datos}: Los datos son gestionados por la Universidad de Burgos, dentro de su marco legal, cumpliéndose con el \acrshort{RGPD}.
	\item \textbf{Licencias}: Todas las herramientas utilizadas en el desarrollo del proyecto son de código abierto o están cubiertas por licencias institucionales de la Universidad de Burgos, salvo el componente \acrshort{AOP}, que requiere licencia comercial si se supera el límite gratuito.
	
	En cuanto a la aplicación \textbf{GeNomIn}, se ha publicado en un repositorio público en GitHub bajo la \gls{Licencia MIT}, reconocida por la Open Source Initiative (OSI). Esta licencia permite:
	\begin{itemize}
		\item Uso libre del software, tanto personal como comercial.
		\item Modificación y redistribución del código.
		\item Conservación de la autoría original.
	\end{itemize}	
	
	La licencia se ha aplicado mediante la inclusión del archivo LICENSE en el repositorio, y se ha indicado en la documentación (README.md). No se requiere registro ni pago para aplicar esta licencia, lo que garantiza la viabilidad legal y económica del proyecto.
\end{itemize}
\subsection{Viabilidad de Implantanción}
\begin{itemize}
	\item \textbf{Entorno Local}: La Universidad dispone de las herramientas y capacidad para la instalación, con licencias de Oracle.
	\item \textbf{Escalabilidad}: como ya se ha demostrado al escalarlo al entorno de Oracle Cloud, no habría problema en el uso de servidores de la propia red.
	\item \textbf{Mantenimiento}: Las tecnologías usadas pueden ser mantenidas por el personal técnico interno.
\end{itemize}
\subsection{Conclusiones} La aplicación web \textbf{GeNomIn}, es viable tanto técnica como legalmente, con un coste contenido y un despliegue seguro en los servidores de la Universidad, cumpliendo con el objetivo inicial de facilitar el trabajo al Servicio en cargado de la gestión de contratos con cargo a proyectos de Investigación.


\apendice{Requisitos y Casos de Uso}
\section{Introducción}
En este anexo B, se presentarán los requisitos de la aplicación, que definen su comportamiento, detallándolo a través de tablas y diagramas para facilitar su comprensión.
Sin perder de vista el objetivo principal de esta aplicación  \textbf{GeNomIn}, que es la de ofrecer una alternativa a al uso de una hoja de cálculo para la gestión de pagos en contratos vinculados a proyectos de investigación, vamos a ir desgranando cada una de las actuaciones para llegar a este objetivo.

Como podemos ver a la siguiente imagen siguiente, en el proceso de la realización de un contrato habría tres actores principales; \textbf{Investigador}, que es quien realiza la solicitud de contratación, \textbf{Solicitante}, que son las personas que presentan una solicitud para participar y los actores principales de la aplicación, los \textbf{Gestores} de \textbf{GeNomIn}, que son los que realizan todas las acciones en la aplicación.
\imagen{secuence}{Diagrama de secuencia del proceso General}

Así, una vez realizada la solicitud y resuelta la convocatoria se realizan el contrato con el solicitante seleccionado y es aquí cuando se iniciaba el traslado de datos a la tabla de Excel, y que ahora, el \textbf{Gestor}, realizará a través de la aplicación y reflejamos en los requisitos, los casos de uso y test ejecutados siguientes.
\Needspace{20\baselineskip}
\section{Requisitos Funcionales}
\subsection{RF01:} El Gestor debe poder acceder al sistema, con su \textbf{usuario} y \textbf{\textbf{contraseña}}
\begin{table}[H]
	\centering
	\renewcommand{\arraystretch}{1.3} 
	\begin{tabularx}{\textwidth}{|l|X|}
		\hline
		\textbf{Identificador} & US-01 \\
		\hline
		\textbf{Título} & Inicio de sesión con usuario ''user01´´ y contraseña ''user01´´ \\
		\hline
		\textbf{Requisito vinculado} & R.F.01 \\
		\hline
		\textbf{Precondiciones} & El usuario ''user01´´ ha sido registrado en el sistema. \\
		\hline
		\textbf{Postcondiciones} & El usuario accede a su cuenta de usuario. \\
		\hline
		\textbf{Proceso} & El usuario se loguea en el sistema. \\
		\hline
		\textbf{Datos de prueba} & Usuario: ''user01´´ Contraseña: ''user01´´ \\
		\hline
		\textbf{Resultados esperados} & El usuario accede a la página de inicio de la aplicación correctamente. \\
		\hline
		\textbf{Estado} & Realizado: Test: t01\_login.js \\
		\hline
	\end{tabularx}
	\caption{Caso de prueba US-01}
	\label{tab:caso_us01}
\end{table}
\Needspace{20\baselineskip}
\subsection{RF02:} Un usuario sin estar dado de alta no puede acceder al sistema.

\begin{table}[H]
	\centering
	\renewcommand{\arraystretch}{1.3} 
	\begin{tabularx}{\textwidth}{|l|X|}
		\hline
		\textbf{Identificador} & US-02 \\
		\hline
		\textbf{Título} & Inicio de sesión con usuario ''no existe'' y contraseña ''ninguna'' \\
		\hline
		\textbf{Requisito vinculado} & R.F.02 \\
		\hline
		\textbf{Precondiciones} & El usuario ''no existe'' no ha sido registrado en el sistema. \\
		\hline
		\textbf{Postcondiciones} & El usuario no accede a su cuenta de usuario. \\
		\hline
		\textbf{Proceso} & El usuario intenta loguearse con usuario y contraseña erróneos \\
		\hline
		\textbf{Datos de prueba} & Usuario: ''no existe'' Contraseña: ''ninguna'' \\
		\hline
		\textbf{Resultados esperados} & El usuario no accede a la página de inicio de la aplicación correctamente y se presenta rechazado \\
		\hline
		\textbf{Estado} & Realizado: Test: t02\_loginfail.js \\
		\hline
	\end{tabularx}
	\caption{Caso de prueba US-02}
	\label{tab:caso_us02}
\end{table}
\Needspace{20\baselineskip}
\subsection{RF03, RF04, RF05 y RF06:} El Gestor accede a con su usuario y contraseña y puede desplegar los distintos menús de la aplicación.
\begin{table}[H]
	\centering
	\renewcommand{\arraystretch}{1.3} 
	\begin{tabularx}{\textwidth}{|l|X|}
		\hline
		\textbf{Identificador} & US-03, US-04, US-05 y US-06 \\
		\hline
		\textbf{Título} & El Gestor accede al sistema y despliega distintos menús \\
		\hline
		\textbf{Requisito vinculado} & RF03, RF04, RF06 y RF06 \\
		\hline
		\textbf{Precondiciones} & El usuario ''user01´´ ha sido registrado en el sistema.\\
		\hline
		\textbf{Postcondiciones} & El Usuario accede a su cuenta. \\
		\hline
		\textbf{Proceso} & El usuario se loguea y despliega Administración, Responsables, Proyectos, Solicitantes\\
		\hline
		\textbf{Datos de prueba} & usuario: ''user01´´ Contraseña: ''user01´´ \\
		\hline
		\textbf{Resultados esperados} & El usuario accede a la página de inicio de la aplicación y despliega correctamente los distintos menús \\
		\hline
		\textbf{Estado} & Realizado: Test: t03\_menuAdmin.js, t04\_logyresp.js, t05\_logypro.js, t06\_logysol.js\\
		\hline
	\end{tabularx}
	\caption{Caso de prueba US-03/06}
	\label{tab:caso_us03}
\end{table}
\Needspace{20\baselineskip}
\subsection{RF07:} El Gestor accede a con su usuario y contraseña y puede crear un nuevo responsable.
\begin{table}[H]
	\centering
	\renewcommand{\arraystretch}{1.3} 
	\begin{tabularx}{\textwidth}{|l|X|}
		\hline
		\textbf{Identificador} & US-07 \\
		\hline
		\textbf{Título} & El Gestor accede y crea uno nuevo responsable \\
		\hline
		\textbf{Requisito vinculado} & R.F.07\\
		\hline
		\textbf{Precondiciones} & El usuario ''user01´´ ha sido registrado en el sistema.\\
		\hline
		\textbf{Postcondiciones} & Debe haber un responsable más \\
		\hline
		\textbf{Proceso} & El usuario se loguea y despliega Responsables pulsa en crea y añade el nuevo responsable\\
		\hline
		\textbf{Datos de prueba} & usuario: ''user01´´ Contraseña: ''user01´´, DNI: 00000005E, Pérez López, Juan \\
		\hline
		\textbf{Resultados esperados} & El usuario accede a la página de inicio de la aplicación correctamente y despliega Responsable y crea uno nuevo. Debe haber una fila más en la tabla de responsables. \\
		\hline
		\textbf{Estado} & Realizado: Test: t07\_crearResp.js\\
		\hline
	\end{tabularx}
	\caption{Caso de prueba US-07}
	\label{tab:caso_us07}
\end{table}
\Needspace{20\baselineskip}
\subsection{RF08:} El Gestor accede a con su usuario y contraseña y puede borrar un responsable.
\begin{table}[H]
	\centering
	\renewcommand{\arraystretch}{1.3} 
	\begin{tabularx}{\textwidth}{|l|X|}
		\hline
		\textbf{Identificador} & US-08 \\
		\hline
		\textbf{Título} & El Gestor accede y borra el responsable creado \\
		\hline
		\textbf{Requisito vinculado} & R.F.08 \\
		\hline
		\textbf{Precondiciones} & El usuario ''user01´´ ha sido registrado en el sistema.\\
		\hline
		\textbf{Postcondiciones} & Debe haber un responsable menos \\
		\hline
		\textbf{Proceso} & El usuario se loguea y accede al menú de responsables, pulsa en el campo de búsqueda e introduce el NIF: 00000005E, pulsa en el lapicero y luego en el botón Delete.\\
		\hline
		\textbf{Datos de prueba} & usuario: ''user01´´ Contraseña: ''user01´´, DNI: 00000005E \\
		\hline
		\textbf{Resultados esperados} & El usuario accede a la página de inicio de la aplicación correctamente y despliega Responsable y borra el responsable creado. Vuelve a buscar por DNI, debe obtener el mensaje Data not found. \\
		\hline
		\textbf{Estado} & Realizado: Test: t08\_borrarResp.js\\
		\hline
	\end{tabularx}
	\caption{Caso de prueba US-08}
	\label{tab:caso_us08}
\end{table}
\Needspace{20\baselineskip}
\subsection{RF09:} El Gestor accede a con su usuario y puede crear una convocatoria.
\begin{table}[H]
	\centering
	\renewcommand{\arraystretch}{1.3} 
	\begin{tabularx}{\textwidth}{|l|X|}
		\hline
		\textbf{Identificador} & US-09 \\
		\hline
		\textbf{Título} & El Gestor accede con su usuario y crea una Convocatoria \\
		\hline
		\textbf{Requisito vinculado} & R.F.09 \\
		\hline
		\textbf{Precondiciones} & El usuario ''user01´´ ha sido registrado en el sistema.\\
		\hline
		\textbf{Postcondiciones} & Debe haber una convocatoria nueva más \\
		\hline
		\textbf{Proceso} & El usuario se loguea accede a convocatorias y crea una nueva \\
		\hline
		\textbf{Datos de prueba} & usuario: ''user01´´ Contraseña: ''user01´´, Convocatoria: PRUEBA PARA BORRAR, Tit: Grado, abierta  \\
		\hline
		\textbf{Resultados esperados} & El usuario accede a la página de inicio de la aplicación correctamente y despliega Convocatorias-Mantenimiento y crea la convocatoria. Debe haber una fila más en la tabla \\
		\hline
		\textbf{Estado} & Realizado: Test: t09\_crearConv.js\\
		\hline
	\end{tabularx}
	\caption{Caso de prueba US-09}
	\label{tab:caso_us09}
\end{table}

\Needspace{20\baselineskip}
\subsection{RF10:} El Gestor accede a con su usuario y contraseña y puede borrar una convocatoria.
\begin{table}[H]
	\centering
	\renewcommand{\arraystretch}{1.3} 
	\begin{tabularx}{\textwidth}{|l|X|}
		\hline
		\textbf{Identificador} & US-10 \\
		\hline
		\textbf{Título} & El Gestor accede y borra la convocatoria de prueba \\
		\hline
		\textbf{Requisito vinculado} & R.F.10 \\
		\hline
		\textbf{Precondiciones} & El usuario ''user01´´ ha sido registrado en el sistema y no hay ninguna convocatoria más abierta.\\
		\hline
		\textbf{Postcondiciones} & No Debe haber una convocatorias disponibles. \\
		\hline
		\textbf{Proceso} & El usuario se loguea y accede a convocatorias, pulsa en el icono del lápiz de la convocatoria a eliminar y pulsa en el botón Delete.\\
		\hline
		\textbf{Datos de prueba} & usuario: ''user01´´ Contraseña: ''user01´´, Convocatoria: PRUEBA PARA BORRAR \\
		\hline
		\textbf{Resultados esperados} & El usuario accede a la página de inicio de la aplicación correctamente y despliega Convocatorias-Mantenimiento y borra la convocatoria. Sale el mensaje Data not found, ya que no quedan más filas. \\
		\hline
		\textbf{Estado} & Realizado: Test: t10\_borrarConvSimple.js\\
		\hline
	\end{tabularx}
	\caption{Caso de prueba US-10}
	\label{tab:caso_us10}
\end{table}

\Needspace{20\baselineskip}
\subsection{RF11:} No debe poderse asociar un solicitante a una convocatoria cerrada.
\begin{table}[H]
	\centering
	\renewcommand{\arraystretch}{1.3} 
	\begin{tabularx}{\textwidth}{|l|X|}
		\hline
		\textbf{Identificador} & US-11 \\
		\hline
		\textbf{Título} & El Gestor accede e intenta crea un solicitante e intenta asociarlo a convocatoria cerrada\\
		\hline
		\textbf{Requisito vinculado} & R.F.11 \\
		\hline
		\textbf{Precondiciones} & El usuario ''user01´´ debe haber una convocatoria cerrada para poder probar.\\
		\hline
		\textbf{Postcondiciones} & No se asocia el solicitante a ninguna convocatoria. \\
		\hline
		\textbf{Proceso} & El usuario se loguea he intenta asociar un solicitante a una convocatoria cerrada\\
		\hline
		\textbf{Datos de prueba} & usuario: ''user01´´ Contraseña: ''user01´´,solicitante: “00001FAKE”, Pérez López, Juan, Convocatoria: Prueba convocatoria 3 cerrada
		 \\
		\hline
		\textbf{Resultados esperados} & El usuario accede a la página de inicio de la aplicación correctamente y despliega Convocatorias-Solicitantes crea el solicitante, pero sin asignar convocatoria \\
		\hline
		\textbf{Estado} & Realizado: Test: t11\_crearSol\_Con\_Close.js\\
		\hline
	\end{tabularx}
	\caption{Caso de prueba US-11}
	\label{tab:caso_uso11}
\end{table}

\Needspace{20\baselineskip}
\subsection{RF12:} No puede crear y asociarse un solicitante a una convocatoria sin tener la titulación adecuada.
\begin{table}[H]
	\centering
	\renewcommand{\arraystretch}{1.3} 
	\begin{tabularx}{\textwidth}{|l|X|}
		\hline
		\textbf{Identificador} & US-12 \\
		\hline
		\textbf{Título} & Inicio de sesión con usuario "user01" y contraseña "user01" y crear solicitante e intentar asociar a convocatoria sin tener la titulación cerrada\\
		\hline
		\textbf{Requisito vinculado} & R.F.12 \\
		\hline
		\textbf{Precondiciones} & El usuario ''user01´´ debe haber una convocatoria para poder probar.\\
		\hline
		\textbf{Postcondiciones} & No se asocia el solicitante a ninguna convocatoria. \\
		\hline
		\textbf{Proceso} & Loguea el usuario y accede e intenta asignar una convocatoria con titulación superior a la que posee\\
		\hline
		\textbf{Datos de prueba} & usuario: ''user01´´ Contraseña: ''user01´´,Solicitante: “00001FAKE”, Pérez López, Juan -Diplomado, Convocatoria: prueba de convocatoria 1- grado\\
		\hline
		\textbf{Resultados esperados} & El usuario accede a la página de inicio de la aplicación correctamente y despliega Convocatorias-Solicitantes edita el solicitante, pero sin asignar convocatoria por la titulación incorrecta. \\
		\hline
		\textbf{Estado} & Realizado: Test: t12\_modSol\_No\_Tit.js\\
		\hline
	\end{tabularx}
	\caption{Caso de prueba US-12}
	\label{tab:caso_uso12}
\end{table}

\Needspace{20\baselineskip}
\subsection{RF13:} Se debe poder asociar un solicitante con titulación adecuada a una solicitud.
\begin{table}[H]
	\centering
	\renewcommand{\arraystretch}{1.3} 
	\begin{tabularx}{\textwidth}{|l|X|}
		\hline
		\textbf{Identificador} & US-13 \\
		\hline
		\textbf{Título} & Inicio de sesión con usuario "user01" y contraseña "user01" y modificar solicitante e intentar asociar a convocatoria con tit. correcta\\
		\hline
		\textbf{Requisito vinculado} & R.F.13 \\
		\hline
		\textbf{Precondiciones} & El usuario ''user01´´ debe haber una convocatoria para poder probar y solicitante con titulación adecuada.\\
		\hline
		\textbf{Postcondiciones} & Se le asocia al solicitante la convocatoria. \\
		\hline
		\textbf{Proceso} & Una vez logueado el usuario selecciona Solicitantes y pulsa en el lápiz del usuario a modificar la convocatoria adecuada y lo asigna\\
		\hline
		\textbf{Datos de prueba} & usuario: ''user01´´ Contraseña: ''user01´´,Solicitante: “00001FAKE”, Pérez López, Juan Convocatoria: CONV017
		\\
		\hline
		\textbf{Resultados esperados} & Al solicitante Juan, se le asigna la convocatoria: CONVO017 \\
		\hline
		\textbf{Estado} & Realizado: Test: t13\_modSol\_OK\_Tit.js\\
		\hline
	\end{tabularx}
	\caption{Caso de prueba US-13}
	\label{tab:caso_uso13}
\end{table}

\Needspace{20\baselineskip}
\subsection{RF14:} Se debe poderse borrar un solicitante existente.
\begin{table}[H]
	\centering
	\renewcommand{\arraystretch}{1.3} 
	\begin{tabularx}{\textwidth}{|l|X|}
		\hline
		\textbf{Identificador} & US-14 \\
		\hline
		\textbf{Título} &Borra el solicitante creado\\
		\hline
		\textbf{Requisito vinculado} & R.F.14 \\
		\hline
		\textbf{Precondiciones} &  Debe existir un solicitante para borrar.\\
		\hline
		\textbf{Postcondiciones} & Se elimina el solicitante. \\
		\hline
		\textbf{Proceso} & Accedemos a la página de inicio de sesión log, se accede a solicitantes y se pulsa en botón editar del solicitante, luego se pulsa en borrar y se confirma.\\
		\hline
		\textbf{Datos de prueba} & usuario: ''user01´´ Contraseña: ''user01´´,Solicitante: “00001FAKE”, Pérez López
		\\
		\hline
		\textbf{Resultados esperados} & El usuario accede a la página de inicio de la aplicación correctamente y despliega Convocatorias-Solicitantes edita el solicitante, y elimina el solicitante antes creado, mostrando el mensaje Action Processed. \\
		\hline
		\textbf{Estado} & Realizado: Test: t14\_delSolicitante.js\\
		\hline
	\end{tabularx}
	\caption{Caso de prueba US-14}
	\label{tab:caso_uso14}
\end{table}

\Needspace{20\baselineskip}
\subsection{RF15, RF16 y RF17:} Se debe poderse crear un contrato para un solicite asignado correctamente a una convocatoria, siempre y cuando las fechas estén dentro del periodo del proyecto. Deben generarse las nóminas correspondientes
\begin{table}[H]
	\centering
	\renewcommand{\arraystretch}{1.3} 
	\begin{tabularx}{\textwidth}{|l|X|}
		\hline
		\textbf{Identificador} & US-15, US-16 y US-17 \\
		\hline
		\textbf{Título} &Crear un contrato a un solicitante	\\
		\hline
		\textbf{Requisito vinculado} & RF.15 \\
		\hline
		\textbf{Precondiciones} & Debe existir solicitante asignado a una convocatoria y que el contrato se encuentre entre las fechas del proyecto.\\
		\hline
		\textbf{Postcondiciones} & Se crea el contrato y se generan las nóminas correspondientes. \\
		\hline
		\textbf{Proceso} & El usuario accede a la página de inicio de la aplicación correctamente y despliega Contratos-Crear, ingresa los datos solicitados y se crea el contrato. \\
		\hline
		\textbf{Datos de prueba} & usuario: ''user01´´ Contraseña: ''user01´´Contratado: Juncal Álvarez Leal
		Fechas: 01-NOV-2025 a 31-OCT-2027, Ret total:  49000, mes: 1500, ss: 500, indemnización: 1000, reserva: 49000, observaciones: PRUEBA TEST
		\\
		\hline
		\textbf{Resultados esperados} & Debe haberse creado el contrato en la tabla correspondiente y 25 nóminas para ese DNI, en la tabla nómina \\
		\hline
		\textbf{Estado} & Realizado: Test: t15\_crearContrato.js, t16\_valFechasContrato.js y t17\_compNominas.js \\
		\hline
	\end{tabularx}
	\caption{Caso de prueba US-15, US-16 y US-17}
	\label{tab:caso_uso15}
\end{table}

\Needspace{20\baselineskip}
\subsection{RF18, RF19 y RF20:} Se debe poder renovar un contrato, siempre y cuando la fecha fin no exceda del fin del proyecto. Además deben añadirse las nóminas correspondientes.
\begin{table}[H]
	\centering
	\renewcommand{\arraystretch}{1.3} 
	\begin{tabularx}{\textwidth}{|l|X|}
		\hline
		\textbf{Identificador} & US-18, US-19 y US-20 \\
		\hline
		\textbf{Título} & Renovación de un contrato para el solicitante, se validan las fechas de la nueva renovación y se actualiza la fecha fin del contrato y las nuevas cantidades de nómina, seg. Social e indemnización	\\
		\hline
		\textbf{Requisito vinculado} & RF.18, RF.19, RF20 \\
		\hline
		\textbf{Precondiciones} & Debe existir solicitante con contrato que no haya llegado a la fecha fin del proyecto.\\
		\hline
		\textbf{Postcondiciones} & Se efectúa la renovación del contrato modificando su fecha fin y se generan las nóminas nuevas nóminas correspondientes. \\
		\hline
		\textbf{Proceso} & El usuario accede a la página de inicio de la aplicación correctamente y despliega Contratos selecciona el contrato a renovar e ingresa los nuevos datos solicitados. \\
		\hline
		\textbf{Datos de prueba} & usuario: ''user01´´ Contraseña: ''user01´´Contrato: CONT123, Fecha fin nueva: 30-sep-2026
		\\
		\hline
		\textbf{Resultados esperados} & El contrato indicado tiene que haberse renovado hasta la fecha indicada con las nuevas cantidades, además de incluirse en Observaciones: RENOVADO. Se deben haber generado 16 nuevas nóminas adicionales. \\
		\hline
		\textbf{Estado} & Realizado: Test: t18\_renovarContratoFechas copy.js, t19\_renovarContrato.js, t20\_compRenNominas.js \\
		\hline
	\end{tabularx}
	\caption{Caso de prueba US-18, US-19 y US-20}
	\label{tab:caso_uso18}
\end{table}

\Needspace{20\baselineskip}
\subsection{RF21, RF22 y RF23:} Se debe poder renunciar a un contrato, siempre y cuando la fecha no sea posteriar al fin del mismo y eliminando las nóminas sobrantes.
\begin{table}[H]
	\centering
	\renewcommand{\arraystretch}{1.3} 
	\begin{tabularx}{\textwidth}{|l|X|}
		\hline
		\textbf{Identificador} & US-21, US-22 y US-23 \\
		\hline
		\textbf{Título} &Renuncia del contrato CONT141 con fecha 31-12-2025, eliminando nóminas y fecha fin.	\\
		\hline
		\textbf{Requisito vinculado} & RF.21, RF.22, RF23 \\
		\hline
		\textbf{Precondiciones} & Debe existir solicitante con contrato en vigor.\\
		\hline
		\textbf{Postcondiciones} & Se efectúa la renuncia del contrato modificando su fecha fin y se eliminan las nóminas correspondientes. \\
		\hline
		\textbf{Proceso} & El usuario accede a la página de inicio de la aplicación correctamente y despliega Contratos selecciona el contrato a renunciar e indica la nueva fecha fin correcta. \\
		\hline
		\textbf{Datos de prueba} & usuario: ''user01´´ Contraseña: ''user01´´, DNI =12345678M, fecha 21-12-2025 \\
		\hline
		\textbf{Resultados esperados} & El contrato indicado tiene que haberse reducido hasta la fecha indicada, además de incluirse en Observaciones: RENUNCIA. Se deben haber eliminado las nóminas sobrantes. \\
		\hline
		\textbf{Estado} & Realizado: Test: t21\_renunciaContrato.js, t22\_renunciaContratoFechas.js, t23\_compRenunnNominas.js \\
		\hline
	\end{tabularx}
	\caption{Caso de prueba US-21, US-22 y US-23}
	\label{tab:caso_uso21}
\end{table}

\Needspace{20\baselineskip}
\subsection{RF24:} Se debe poder genera un informe de la nómina de un mes determinado y luego imprimirla en PDF.
\begin{table}[H]
	\centering
	\renewcommand{\arraystretch}{1.3} 
	\begin{tabularx}{\textwidth}{|l|X|}
		\hline
		\textbf{Identificador} & US-24 \\
		\hline
		\textbf{Título} &Comprobación si se genera informe de la nómina de un mes determinado y se puede imprimir dicho mes pdf.	\\
		\hline
		\textbf{Requisito vinculado} & RF.24 \\
		\hline
		\textbf{Precondiciones} & Deben existir contratos en vigor para el mes/años solicitados.\\
		\hline
		\textbf{Postcondiciones} & Se genera el informe en PDF \\
		\hline
		\textbf{Proceso} & El usuario accede a la sección de informes Nómina-mes, selecciona un mes y un año y pulsa “consultar”. 	Una vez listado podrá escoger filtrar por orgánica y volviendo a pulsar “consultar”
		Pulsando Generar PDF descarga el informe personalizado.
		 \\
		\hline
		\textbf{Datos de prueba} & usuario: ''user01´´ Contraseña: ''user01´´, Mes: “Diciembre” Año “2025” \\
		\hline
		\textbf{Resultados esperados} & Se lista los datos de nómina correspondientes al mes y año. Se puede realizar filtro por orgánica descargar un pdf personalizado. \\
		\hline
		\textbf{Estado} & Realizado: Test: t24\_infNominaMesyPDF.js \\
		\hline
	\end{tabularx}
	\caption{Caso de prueba US-24}
	\label{tab:caso_uso24}
\end{table}

\Needspace{20\baselineskip}
\subsection{RF25:} Se debe poder comprobar los vencimientos de contratos entre fechas.
\begin{table}[H]
	\centering
	\renewcommand{\arraystretch}{1.3} 
	\begin{tabularx}{\textwidth}{|l|X|}
		\hline
		\textbf{Identificador} & US-25 \\
		\hline
		\textbf{Título} &Comprobación de vencimientos de contratos entre fechas.	\\
		\hline
		\textbf{Requisito vinculado} & RF.25 \\
		\hline
		\textbf{Precondiciones} & Deben existir contratos en vigor entre las fechas solicitadas.\\
		\hline
		\textbf{Postcondiciones} & Se visualiza el informe \\
		\hline
		\textbf{Proceso} & El usuario accede a la sección de informes Vencimientos y selecciona intervalo de fechas y pulsa botón Consulta
		\\
		\hline
		\textbf{Datos de prueba} & usuario: ''user01´´ Contraseña: ''user01´´, 01-01-2020 a 31-12-2025 \\
		\hline
		\textbf{Resultados esperados} & Se listan los datos de vencimientos de contrato en el periodo solicitado. \\
		\hline
		\textbf{Estado} & Realizado: Test: t25\_infVencimientos.js \\
		\hline
	\end{tabularx}
	\caption{Caso de prueba US-25}
	\label{tab:caso_uso25}
\end{table}

\Needspace{20\baselineskip}
\subsection{RF26:} Se debe poder obtener un listado con todos los datos generales de contratos y aplicar filtros.
\begin{table}[H]
	\centering
	\renewcommand{\arraystretch}{1.3} 
	\begin{tabularx}{\textwidth}{|l|X|}
		\hline
		\textbf{Identificador} & US-26 \\
		\hline
		\textbf{Título} &Listado de contratos. \\
		\hline
		\textbf{Requisito vinculado} & RF.26 \\
		\hline
		\textbf{Precondiciones} & Deben existir contratos en vigor.\\
		\hline
		\textbf{Postcondiciones} & Se visualiza el informe \\
		\hline
		\textbf{Proceso} & El usuario accede a la sección de informes Listado de contratos. \\
		\hline
		\textbf{Datos de prueba} & usuario: ''user01´´ Contraseña: ''user01´´ \\
		\hline
		\textbf{Resultados esperados} & Se listan los datos de vencimientos de contrato en el periodo solicitado. \\
		\hline
		\textbf{Estado} & Realizado: Test: t26\_listaContratosFull.js \\
		\hline
	\end{tabularx}
	\caption{Caso de prueba US-26}
	\label{tab:caso_uso26}
\end{table}

\Needspace{20\baselineskip}
\subsection{RF27:} Se debe poder obtener un listado con las nóminas de un contratado en un periodo indicado. Debe ofrecer el total de remuneración en ese periodo.
\begin{table}[H]
	\centering
	\renewcommand{\arraystretch}{1.3} 
	\begin{tabularx}{\textwidth}{|l|X|}
		\hline
		\textbf{Identificador} & US-27 \\
		\hline
		\textbf{Título} &Listado de nóminas de contratado por periodo \\
		\hline
		\textbf{Requisito vinculado} & RF.27 \\
		\hline
		\textbf{Precondiciones} & Deben existir contratos en vigor para un solicitante en el periodo.\\
		\hline
		\textbf{Postcondiciones} & Se visualiza el informe \\
		\hline
		\textbf{Proceso} &El usuario accede a la sección de informes Nóminas-periodo. Selecciona un contratado de la selección e indica un intervalo de fechas y pulsa buscar. \\
		\hline
		\textbf{Datos de prueba} & usuario: ''user01´´ Contraseña: ''user01´´ Contratado: Juncal Álvarez Pérez, periodo: 01-01-2025 a 31-12-2025\\
		\hline
		\textbf{Resultados esperados} & Se ofrece un listado de las nóminas de ese contratado en el periodo solicitado junto con su total para comprobación = 4600.25 \\
		\hline
		\textbf{Estado} & Realizado: Test: t27\_infNomPeriodo.js \\
		\hline
	\end{tabularx}
	\caption{Caso de prueba US-27}
	\label{tab:caso_uso27}
\end{table}
\Needspace{20\baselineskip}
En la siguiente tabla vemos a través del informe de Allure, los test pasados (informe completo \href{https://far0010.github.io/TFGUBU-Fran_Arroyo/informe/#} {aquí})
El código de los test en \acrshort{JS} pueden verse: \href{project-docs/memoria/test}{aquí}
\imagen{testTodos}{Informe Allure de Test}
\apendice{Diseño GeNomIn}
\section{Introducción}
En este Anexo C, voy a explicar como se ha realizado la aplicación \textbf{GeNomIn} a través de la plataforma \acrfull{APEX} 2402.
Se datallará como se han implementado las diferentes páginas para obtener las funcionalidades y requisitos exigidos por los usuarios.

Como ya se indicó en el \textbf{apartado 4.2} de la memoria, una vez instalado \acrshort{APEX} y la base de datos \acrfull{ODB}, es preciso crear los usuarios que utilizarán la aplicación, en este caso a parte del \textbf{Administrador}, se crearon un \textbf{desarrollador} y un \textbf{usuario} final.

\section{App Builder}
La herramienta para construcción de aplicaciones de apex ~\cite{OracleAPEXAdministration}, es un acrshort{IDE} con poco código, a través de páginas y componentes compartidos (\textbf{shared components}).
Inicialmente podemos crear nuestra apliación a través de varias fuentes; con ficheros (XML, CSV XLSX, JSON, SQL), copiando aplicaciones existentes o directamente con el asistente, que nos creará una página de lógin y nos irá permitiendo añadir páginas y componentes para personalizar nuestro diseño.
\imagen{appbuilder}{Creación de app APEX}

En los siguientes apartados se irán detallando los aspectos fundamentales para el desarrollo de las páginas necesarias e importantes en este proyecto y el código asociado, ya sea en \acrshort{SQL}, \acrshort{PL/SQL} o \acrshort{JS}. Todo este código también está disponible en el repositorio \href{https://github.com/far0010/TFGUBU-Fran_Arroyo/tree/main/project-docs/memoria/sql}{Código de GeNomIn}

\section{Página Proyectos}
En esta página se detallan todos los proyectos que existen en la base de datos a través de una consulta simple \acrshort{SQL}. El gestor de \textbf{acciones} de \acrshort{APEX}, permite ejecutar búsquedas y diversos filtros, además de añadir, eliminar registros y actualizar datos.
\begin{lstlisting}[language=SQL, caption={Informe de todos los Proyectos}]
	SELECT 
	p.ORGANICA, 
	p.TITULO, 
	r.NOMBRE || ' ' || r.ape1 || ' ' || r.ape2 AS REPONSABLE,
	p.fecha_ini, 
	p.fecha_fin  
	FROM proyecto p
	JOIN responsable r ON p.responsable = r.dni;
\end{lstlisting}

\section{Página Responsables}
Al igual que la página anterior usa una consulta \acrshort{SQL}, para mostrar todos los responsables, tengan o no, proyectos abiertos. También permite realizar \textbf{Acciones}.
\begin{lstlisting}[language=SQL, caption={Informe de todos los Responsables}]
	SELECT 
	R."DNI", 
	R."APE1",
	R."APE2",
	R."NOMBRE",
	D."NOMBRE" AS "DEPARTAMENTO"
	FROM RESPONSABLE R
	JOIN DEPTOS D
	ON R."DEPTO" = D."REF";
\end{lstlisting}

\section{Página Mantenimiento Convocatorias}
En esta página se hace una consulta \acrshort{SQL} que devuelve el detalle de todas las convocatorias y su responsable asociado, incluyendo \textbf{Acciones}.
\begin{lstlisting}[language=SQL, caption={Informe de todas las Convocatorias}]
	SELECT C.REFERENCIA,
	C.TITULO,
	C.TIT_REQUERIDA,
	CASE C.ABIERTO WHEN 1 THEN 'S\'{i}' ELSE 'No' END AS ABIERTO,
	R.APE1 || ' ' || R.APE2 || ', ' || R.NOMBRE AS INVESTIGADOR,
	C.REF_PROY,
	C.NUM_PLAZAS
	FROM CONVOCATORIA C
	LEFT JOIN RESPONSABLE R ON C.REF_INV = R.DNI
\end{lstlisting}
\subsection{Crear Convocatoria}
Para crear una nueva convocatoria ha sido preciso el uso de; 
primero, una consulta \acrshort{SQL}, que no muestra todos los investigadores:
\begin{lstlisting}[language=SQL, caption={Presenta de todos las Investigadores}]
	SELECT APE1 || ' ' || APE2 || ', ' || NOMBRE AS DISPLAY_VALUE, DNI AS RETURN_VALUE
	FROM RESPONSABLE
	ORDER BY APE1, APE2
\end{lstlisting}
segundo, una consulta que una vez escogido el investigador, muestre los proyectos abiertos:
\begin{lstlisting}[language=SQL, caption={Presenta informe proyectos abiertos para el investigador seleccionado}]
	SELECT P.TITULO, P.ORGANICA
	FROM PROYECTO P
	WHERE P.RESPONSABLE = :P9_REF_INV
	AND (P.FECHA_FIN IS NULL OR P.FECHA_FIN >= TRUNC(SYSDATE))
	ORDER BY P.TITULO;
\end{lstlisting}
Y dos acciones dinámicas en \acrshort{JS}, en la que se controla: 
primera, si el investigador tiene proyectos abiertos desde la fecha actual, si no, muestra mensaje de aviso:
\begin{lstlisting}[language=JavaScript, caption={Control y aviso de proyecto abiertos de investigador seleccionado}]
	apex.message.clearErrors();
	
	// Esperamos brevemente a que el combo termine de cargarse
	setTimeout(function() {
		const opciones = $("#P9_REF_PROY option");
		const seleccionado = $v("P9_REF_PROY");
		
		// Si solo hay una opcion la nula o ninguna
		if (opciones.length <= 1 || !opciones[1]?.value) {
			apex.message.alert("AVISO: El investigador seleccionado no tiene proyectos abiertos.");
		}
	}, 300);
\end{lstlisting}
y la segunda, nos muestra el tiempo restante del proyecto para tener en cuenta en la duración de contratos.
\begin{lstlisting}[language=JavaScript, caption={Informa de la duración restante del proyecto elegido}]
	apex.message.clearErrors();
	
	var duracion = $v("P9_DURACION");
	if (duracion && duracion.trim() !== "") {
		apex.message.alert("AVISO: La duracion del proyecto es: " + duracion);
	}
\end{lstlisting}

\section{Página de Solicitantes}
Al igual que las anteriores, muestra una búsqueda \acrshort{SQL}, simple, con todos los solicitantes y si tienen o no una convocatoria asignada:
\begin{lstlisting}[language=SQL, caption={Presenta informe de solicitantes}]
		select "DNI_SOL", 
		"APE1_SOL",
		"APE2_SOL",
		"NOM_SOL",
		"TIT_SOL",
		"REF_CON"
		from "#OWNER#"."SOLICITANTE" 
\end{lstlisting}
\subsection{Crear solicitante}
Para la creación de un nuevos solicitantes, hay que tener en cuenta varios aspectos, como son;
si quedan plazas disponibles en la convocatoria, esto se ha resuelto con una búsqueda \acrshort{SQL} que cuenta los solicitantes de las convocatorias y las muestra si son mayores de 0:
\begin{lstlisting}[language=SQL, caption={Presenta informe de solicitantes}]
	SELECT C.TITULO, C.REFERENCIA
	FROM CONVOCATORIA C
	LEFT JOIN (
	SELECT REF_CON, COUNT(*) AS NUM_SOLICITANTES
	FROM SOLICITANTE
	GROUP BY REF_CON
	) S ON C.REFERENCIA = S.REF_CON
	WHERE NVL(S.NUM_SOLICITANTES, 0) < C.NUM_PLAZAS
	ORDER BY C.TITULO, C.REFERENCIA;
\end{lstlisting}
Luego es preciso verificar si la convocatoria está abierta y que envíe un mensaje de aviso en caso de estar cerrada. Esto se ha resuelto con una primera \textbf{acción dinámica}, con consulta \acrshort{SQL} que guarda el campo abierto en un \textbf{textbox} oculto. Para ello, se utilizan las opciones de la página \acrshort{APEX}: \textbf{ítems to submit} y \textbf{Affected Elements}, que nos permiten enviar y guardar los datos.
\begin{lstlisting}[language=SQL, caption={Comprueba campo abierto de convocatoria}]
	SELECT ABIERTO
	FROM CONVOCATORIA 
	WHERE REFERENCIA = :P7_REF_CON 
\end{lstlisting}
Una vez guardado el resultado en el campo oculto \textbf{P7\_CON\_OPEN}, comprobamos con la siguiente parte de esta primera acción, a través de \acrshort{JS}:
\begin{lstlisting}[language=JavaScript, caption={Control y aviso de convocatoria cerrada}]
	apex.message.clearErrors();
	
	const abierto = $v('P7_CON_OPEN');
	
	if (abierto == '0') {
		const mensaje = 'Esta convocatoria esta cerrada: ';
		apex.message.alert("AVISO: "+mensaje);
		$s('P7_REF_CON', ''); // Limpia la convocatoria para que elijan otra
	} 
\end{lstlisting}
El último paso es para comprobar si el solicitante tiene la titulación que requiere la convocatoria, procediendo de forma similar, con otra \textbf{acción dinámica}. Primero una consulta sencilla \acrshort{SQL} que guarda la titulación requerida en un campo oculto, \textbf{\textbf{P7\_TIT-REQ}}.
\begin{lstlisting}[language=SQL, caption={Comprueba campo de titulación requerida}]
	SELECT TIT_REQUERIDA
	FROM CONVOCATORIA 
	WHERE REFERENCIA = :P7_REF_CON
\end{lstlisting}
y luego mostramos un aviso con \acrshort{JS}, en el caso de que la titulación del solicitante no sea igual o superior a la requerida en la convocatoria.
\begin{lstlisting}[language=JavaScript, caption={Control y aviso de titulación insuficiente}]
	apex.message.clearErrors();
	if ($v('P7_CON_OPEN') == '1'){
		// compruebo que la convocatoria esta abierta
		const valReq = $v('P7_TIT-REQ');
		const valSol = $v('P7_TIT_SOL');
		
		if (valReq !== '' && valSol !== '') {
			const tit_requerido = parseInt(valReq, 10);
			const tit_usuario = parseInt(valSol, 10);
			
			if (tit_usuario < tit_requerido) {
				// Buscar el texto asociado al valor requerido
				const select = document.getElementById('P7_TIT_SOL');
				const textoRequerido = Array.from(select.options).find(opt => opt.value === valReq)?.text || 'nivel superior';
				const mensaje = 'Nivel academico insuficiente. Se requiere al menos: ' + textoRequerido;
				apex.message.alert("AVISO: "+mensaje);
				$s('P7_REF_CON', ''); // Limpia la convocatoria para que elijan otra
			}
		} 
	}
\end{lstlisting}

\section{Página de Contratos}
Este primer \textbf{informe interactivo}, no requiere de consultas adicionales ya que \acrshort{APEX}, realiza esta función son solo indicarle la tabla que queremos mostrar, en este caso CONTRATOS.
\subsection{Página Nuevo Contrato}
Para la realización de un nuevo contrato tendremos que realizar diversas comprobaciones. Inicialmente ofrecemos las convocatorias disponibles, \acrshort{APEX} también nos ofrece esta opción sin necesidad de programación adicional a través de \textbf{List of Values: CONVOCATORIA.TITULO}, en  (shared componentes).
En una primera consulta \acrshort{SQL}, obtendremos los solicitantes que participan en la convocatoria elegida anteriormente:
\begin{lstlisting}[language=SQL, caption={Comprueba campo de titulación requerida}]
	SELECT NOM_SOL || ' ' || APE1_SOL || ' ' || APE2_SOL AS DISPLAY_VALUE,
	DNI_SOL AS RETURN_VALUE
	FROM SOLICITANTE S
	WHERE REF_CON = :P12_CONVOCATORIA
	AND NOT EXISTS (
	SELECT 1
	FROM CONTRATOS C
	WHERE C.CONTRATADO = S.DNI_SOL
	AND C.F_FIN >= SYSDATE -- compruebo si esta activo
	)
	ORDER BY APE1_SOL;
\end{lstlisting}
Y ahora necesitamos tres \textbf{acciones dinámicas}, una que va a cargar las fechas de inicio y fin del proyecto, en campos ocultos, P12\_F\_INI\_OCULTO y P12\_F\_FIN\_OCULTO  a través de dos consultas \acrshort{SQL}:
\begin{lstlisting}[language=SQL, caption={Carga fecha inicio proyecto}]
	SELECT TO_CHAR(P.FECHA_INI, 'YYYY-MM-DD') AS FECHA_INI_PROYECTO
	FROM CONVOCATORIA C 
	JOIN PROYECTO P ON P.ORGANICA = C.REF_PROY
	WHERE C.REFERENCIA = :P12_CONVOCATORIA
\end{lstlisting}

\begin{lstlisting}[language=SQL, caption={Carga fecha fin proyecto}]
	SELECT TO_CHAR(P.FECHA_FIN, 'YYYY-MM-DD') AS FECHA_FIN_PROYECTO
	FROM CONVOCATORIA C 
	JOIN PROYECTO P ON P.ORGANICA = C.REF_PROY
	WHERE C.REFERENCIA = :P12_CONVOCATORIA
\end{lstlisting}

y \acrshort{JS}, para forzar de nuevo la comprobación en caso de error:
\begin{lstlisting}[language=JavaScript, caption={Fuerza nueva petición de fechas si son erróneas}]
	apex.event.trigger('#P12_F_INI', 'change');
	apex.event.trigger('#P12_F_FIN', 'change');
\end{lstlisting}

Una vez cargadas las fechas de inicio y fin de los proyectos es preciso cotejarlas con las introducidas por el usuario, para comprobar si están dentro del margen del proyecto a través de dos \textbf{acciones dinámicas} en \acrshort{JS}:
\begin{lstlisting}[language=JavaScript, caption={Controla que la fecha inicio sea posterior al inicio del proyecto}]
	const fechaContratoStr = $v('P12_F_INI');
	const fechaProyectoStr = $v('P12_F_INI_OCULTO');
	const fechaContratoIni = new Date(fechaContratoStr);
	const fechaProyectoIni = new Date(fechaProyectoStr);
	console.log('INI_OCULTO:', $v('P12_F_INI_OCULTO'));
	console.log('FIN_OCULTO:', $v('P12_F_FIN_OCULTO'));
	
	
	if (!isNaN(fechaContratoIni) && !isNaN(fechaProyectoIni)) {
		if (fechaContratoIni < fechaProyectoIni) {
			const mensaje = 'La fecha de inicio debe ser igual o posterior a: ' +
			fechaProyectoIni.toLocaleDateString('es-ES');
			apex.message.alert("AVISO: " + mensaje);
			$s('P12_F_INI', '');
			$('#P12_F_INI').trigger('change');
		}
	}
\end{lstlisting}

Y ahora comprobamos si la fecha de fin de contrato, es posterior a la de inicio y si sobrepasa al proyecto:
\begin{lstlisting}[language=JavaScript, caption={Controla que la fecha fin sea posterior a inicio y anterior al fin del proyecto}]
	const fechaContratoFinStr = $v('P12_F_FIN');
	const fechaContratoIniStr = $v('P12_F_INI');
	const fechaProyectoFinStr = $v('P12_F_FIN_OCULTO');
	const fechaContratoIni = new Date(fechaContratoIniStr);
	const fechaContratoFin = new Date(fechaContratoFinStr);
	const fechaProyectoFin = new Date(fechaProyectoFinStr);
	
	if (!isNaN(fechaContratoFin) && !isNaN(fechaContratoIni)) {
		if(fechaContratoFin<fechaContratoIni){
			const mensaje1 = 'La fecha de fin debe ser igual o posterior a la de inicio';
			apex.message.alert("AVISO: " + mensaje1);
			$s('P12_F_FIN', $v('P12_F_INI'));
			$('#P12_F_FIN').trigger('change');
		}
	}
	if (!isNaN(fechaContratoFin) && !isNaN(fechaProyectoFin)) {
		if (fechaContratoFin > fechaProyectoFin) {
			const mensaje2 = 'La fecha de fin debe ser igual o anterior a: ' + fechaProyectoFin.toLocaleDateString('es-ES');
			apex.message.alert("AVISO: " + mensaje2);
			$s('P12_F_FIN', '');
			$('#P12_F_FIN').trigger('change');
		}
	}
\end{lstlisting}

Una vez realizadas las comprobaciones, es preciso el paso más importante generar las nóminas para el contrato en la tabla nómina. Para ello se requiere un \textbf{proceso} en \acrshort{PL/SQL}:
\begin{lstlisting}[language=PLSQL, caption={Proceso que genera las nóminas relativas a un nuevo contrato}]
	DECLARE
	v_fecha_ini DATE := :P12_F_INI;
	v_fecha_fin DATE := :P12_F_FIN;
	v_total_meses      PLS_INTEGER;
	v_meses_generados  PLS_INTEGER := 0;
	v_mes_actual       DATE;
	
	v_ret_mes NUMBER := :P12_RET_MES;
	v_ss_mes NUMBER := :P12_SS_MES;
	v_indemnizacion NUMBER := :P12_INDEMNIZACION;
	
	BEGIN
	EXECUTE IMMEDIATE 'ALTER SESSION SET NLS_NUMERIC_CHARACTERS = '',.''';
	
	v_total_meses := MONTHS_BETWEEN(TRUNC(v_fecha_fin, 'MM'), TRUNC(v_fecha_ini, 'MM')) + 1;
	
	WHILE v_meses_generados < v_total_meses LOOP
	v_mes_actual := ADD_MONTHS(TRUNC(v_fecha_ini, 'MM'), v_meses_generados);
	
	INSERT INTO NOMINA (
	dni_nom, mes, anno, mensualidad, seg_soc, observaciones
	) VALUES (
	:P12_CONTRATADO,
	TO_NUMBER(TO_CHAR(v_mes_actual, 'MM')),
	TO_NUMBER(TO_CHAR(v_mes_actual, 'YYYY')),
	v_ret_mes,
	CASE
	WHEN v_meses_generados = 0 THEN 0
	ELSE v_ss_mes
	END,
	NULL
	);
	
	v_meses_generados := v_meses_generados + 1;
	END LOOP;
	
	-- Mes extra con indemnizacion
	v_mes_actual := ADD_MONTHS(TRUNC(v_fecha_ini, 'MM'), v_total_meses);
	
	INSERT INTO NOMINA (
	dni_nom, mes, anno, mensualidad, seg_soc, observaciones
	) VALUES (
	:P12_CONTRATADO,
	TO_NUMBER(TO_CHAR(v_mes_actual, 'MM')),
	TO_NUMBER(TO_CHAR(v_mes_actual, 'YYYY')),
	v_indemnizacion,
	v_ss_mes,
	'indemnizacion '
	);
	
	EXCEPTION
	WHEN OTHERS THEN
	RAISE_APPLICATION_ERROR(-20099, 'Error interno: ' || SQLERRM);
	END;
\end{lstlisting}

\subsection{Página Renovación}
La renovación de un contrato requiere de varios controles. Inicialmente tendremos que seleccionar los contratos con fecha fin posterior a la actual,  a través de una consulta \acrshort{SQL}, en el campo P13\_SEL\_CONT
\begin{lstlisting}[language=SQL, caption={Selección de contratos abiertos}]
	SELECT 
	C.REF_CONTRATO || ' - ' || S.NOM_SOL || ' ' || S.APE1_SOL || ' ' || S.APE2_SOL AS DISPLAY_VALUE,
	C.REF_CONTRATO AS RETURN_VALUE
	FROM CONTRATOS C
	JOIN SOLICITANTE S ON C.CONTRATADO = S.DNI_SOL
	WHERE C.F_FIN >= TRUNC(SYSDATE)
	ORDER BY C.REF_CONTRATO
\end{lstlisting}
Seguidamente a través de una \textbf{acción dinámica} mostraremos los campos de ese contrato seleccionado anteriormente con otra consulta \acrshort{SQL}:
\begin{lstlisting}[language=SQL, caption={Obtención de datos del contrato a renovar}]
	SELECT 
	C.REF_CONTRATO,
	S.NOM_SOL || ' ' || S.APE1_SOL || ' ' || S.APE2_SOL AS NOMBRE_COMPLETO,
	TO_CHAR(C.F_INI, 'DD-MM-YYYY') AS F_INI,
	TO_CHAR(C.F_FIN, 'DD-MM-YYYY') AS F_FIN,
	TO_CHAR(C.RET_MES, '999G999G990D00') AS RET_MES,
	TO_CHAR(C.SS_MES, '999G999G990D00') AS SS_MES,
	TO_CHAR(C.INDEMNIZACION, '999G999G990D00') AS INDEMNIZACION
	FROM CONTRATOS C
	JOIN SOLICITANTE S ON C.CONTRATADO = S.DNI_SOL
	WHERE C.REF_CONTRATO = :P13_SEL_CONT
\end{lstlisting}
Y posteriormente guardaremos dentro de esta misma acción dinámica la fecha fin en un campo oculto P13\_NEW\_FECHA, para su posterior comparación. Como en anteriores ocasiones esto es posible por la propiedad de \acrshort{APEX}, Affected elements.
\begin{lstlisting}[language=SQL, caption={Obtención de fecha fin del contrato a renovar}]
	SELECT P.FECHA_FIN
	FROM CONTRATOS C
	JOIN CONVOCATORIA V ON C.CONVOCATORIA = V.REFERENCIA
	JOIN PROYECTO P ON V.REF_PROY = P.ORGANICA
	WHERE C.REF_CONTRATO = :P13_SEL_CONT
\end{lstlisting}

Una vez obtenidos los datos del contrato en vigor, se introduce la fecha de la renovación, que tiene que comprobarse a través de otra \textbf{acción dinámica} asociada a un botón \textbf{verifica-fecha}, que cambiará a color verde si es correcta y mostrará error en caso contrario. Esta acción dinámica es verificada por dos códigos \acrshort{JS}:
Este primer código es bastante más complejo, ya que ha requerido la conversión de fechas.
\begin{lstlisting}[language=JavaScript, caption={Verifica la fecha de renovación se posterior a la actual y anterior a fin proyecto}]
	//  parsear fechas en formato 'DD-MM-YYYY' o 'DD-MON-YYYY'
	function parseFechaFlexible(fechaStr) {
		if (!fechaStr) {
			console.warn('Fecha vacia o null:', fechaStr);
			return null;
		}
		
		const partes = fechaStr.trim().split('-');
		if (partes.length !== 3) {
			console.error('Formato inesperado:', fechaStr);
			return null;
		}
		
		const dia = parseInt(partes[0], 10);
		const año = parseInt(partes[2], 10);
		
		// Detectar si el mes es numero o texto
		const mesTexto = partes[1].toUpperCase();
		const meses = {
			'ENE': 0, 'FEB': 1, 'MAR': 2, 'ABR': 3,
			'MAY': 4, 'JUN': 5, 'JUL': 6, 'AGO': 7,
			'SEP': 8, 'OCT': 9, 'NOV': 10, 'DIC': 11
		};
		
		let mes;
		if (!isNaN(parseInt(mesTexto, 10))) {
			mes = parseInt(mesTexto, 10) - 1; // formato numerico
		} else {
			mes = meses[mesTexto]; // formato texto
		}
		
		if (mes === undefined || isNaN(dia) || isNaN(año)) {
			console.error(' Problema al interpretar partes de la fecha:', partes);
			return null;
		}
		
		return new Date(año, mes, dia);
	}
	
	// Leer valores 
	const fechaInicioStr         = $v('P13_F_INI');
	const fechaContratoFinStr    = $v('P13_F_FIN');
	const nuevaFechaFinStr       = $v('P13_NEW_F_FIN');
	const fechaProyectoFinStr    = $v('P13_NEW_FECHA');
	
	// Parsear las fechas
	const fechaInicio            = parseFechaFlexible(fechaInicioStr);
	const fechaContratoFin       = parseFechaFlexible(fechaContratoFinStr);
	const nuevaFechaFin          = parseFechaFlexible(nuevaFechaFinStr);
	const fechaProyectoFin       = parseFechaFlexible(fechaProyectoFinStr);
	// semaforo
	let estadoVerificacion = 'ok';
	// nueva fecha debe ser posterior a actual
	if (nuevaFechaFin && fechaContratoFin) {
		if (nuevaFechaFin <= fechaContratoFin) {
			apex.message.alert("AVISO: La nueva fecha debe ser posterior a la actual");
			$s('P13_NEW_F_FIN', '');
			$('#P13_NEW_F_FIN').trigger('change');
			estadoVerificacion = 'fail';
		}
	}
	
	// nueva fecha no debe pasar la fecha fin de proyecto
	if (nuevaFechaFin && fechaProyectoFin) {
		if (nuevaFechaFin > fechaProyectoFin) {
			apex.message.alert("AVISO: La nueva fecha debe ser igual o anterior a: " + fechaProyectoFin.toLocaleDateString('es-ES'));
			$s('P13_NEW_F_FIN', '');
			$('#P13_NEW_F_FIN').trigger('change');
			estadoVerificacion = 'fail';
		}
	}
	
	// cambio color si ok
	const $boton = $('#btn_verifica');
	
	if ($boton.length) {
		// Elimina clases anteriores
		$boton.removeClass('t-Button--success t-Button--hot t-Button--danger');
		
		// Aplica color segUn estado
		if (estadoVerificacion === "ok") {
			$boton.addClass('t-Button--success');
		}
	}
\end{lstlisting}

Como en anteriores casos, si la fecha fuera errónea, habría que reactivar el campo de fecha:
\begin{lstlisting}[language=JavaScript, caption={Reactiva fecha renovación se es errónea}]
	apex.event.trigger('#P13_NEW_F_FIN', 'change');
\end{lstlisting}

Finalmente como en el caso inicial de creación de nóminas, necesitamos ejecutar un \textbf{procedimiento} en \acrshort{PL/SQL}, que genere las nuevas nóminas adicionales:
\begin{lstlisting}[language=PLSQL, caption={Genera las nuevas nóminas tras la renovación}]
	DECLARE
	v_fecha_ini DATE := :P13_F_FIN; 
	v_fecha_fin DATE := :P13_NEW_F_FIN;
	v_total_meses      PLS_INTEGER;
	v_meses_generados  PLS_INTEGER := 0;
	v_mes_actual       DATE;
	
	v_ret_mes NUMBER := :P13_NEW_SAL;
	v_ss_mes NUMBER := :P13_NEW_SS;
	v_indemnizacion NUMBER := :P13_NEW_IND;
	
	v_dni VARCHAR2(9);
	
	v_ultimo_mes DATE;
	
	
	BEGIN
	EXECUTE IMMEDIATE 'ALTER SESSION SET NLS_NUMERIC_CHARACTERS = '',.''';
	
	--  Obtengo el DNI desde la tabla CONTRATO
	SELECT CONTRATADO
	INTO v_dni
	FROM CONTRATOS
	WHERE REF_CONTRATO = :P13_SEL_CONT;
	
	-- Actualizo la nueva fecha fin del contrato e importes anado observacion de renovacion.
	
	UPDATE CONTRATOS
	SET 
	F_FIN = :P13_NEW_F_FIN,
	RET_MES = :P13_NEW_SAL,
	SS_MES = :P13_NEW_SS,
	INDEMNIZACION = :P13_NEW_IND,
	OBSERVACIONES = 'Renovado'
	WHERE REF_CONTRATO = :P13_SEL_CONT;
	
	-- Buscamos el ultimo mes insertado anteriormente
	SELECT TO_DATE(anno || LPAD(mes, 2, '0'), 'YYYYMM')
	INTO v_ultimo_mes
	FROM (
	SELECT mes, anno
	FROM NOMINA
	WHERE dni_nom = v_dni
	ORDER BY anno DESC, mes DESC
	)
	WHERE ROWNUM = 1;
	
	-- Actualizar nueva retribucion del primer nuevo mes
	UPDATE NOMINA
	SET mensualidad = v_ret_mes
	WHERE dni_nom = v_dni
	AND mes = TO_NUMBER(TO_CHAR(v_ultimo_mes, 'MM'))
	AND anno = TO_NUMBER(TO_CHAR(v_ultimo_mes, 'YYYY'));
	
	-- Calculo de nuevos meses a generar
	v_total_meses := MONTHS_BETWEEN(TRUNC(v_fecha_fin, 'MM'), ADD_MONTHS(TRUNC(v_ultimo_mes, 'MM'), 1)) + 1;
	
	-- Insercion nuevos meses con salario y SS
	WHILE v_meses_generados < v_total_meses LOOP
	v_mes_actual := ADD_MONTHS(TRUNC(v_ultimo_mes, 'MM'), v_meses_generados + 1);
	
	INSERT INTO NOMINA (
	dni_nom, mes, anno, mensualidad, seg_soc, observaciones
	) VALUES (
	v_dni,
	TO_NUMBER(TO_CHAR(v_mes_actual, 'MM')),
	TO_NUMBER(TO_CHAR(v_mes_actual, 'YYYY')),
	v_ret_mes,
	v_ss_mes,
	NULL
	);
	
	v_meses_generados := v_meses_generados + 1;
	END LOOP;
	
	-- Insertar el ultimo mes adicional solo con la indemnizacion
	v_mes_actual := ADD_MONTHS(TRUNC(v_ultimo_mes, 'MM'), v_total_meses + 1);
	
	INSERT INTO NOMINA (
	dni_nom, mes, anno, mensualidad, seg_soc, observaciones
	) VALUES (
	v_dni,
	TO_NUMBER(TO_CHAR(v_mes_actual, 'MM')),
	TO_NUMBER(TO_CHAR(v_mes_actual, 'YYYY')),
	v_indemnizacion,
	v_ss_mes,
	'Indemnizacion final'
	);
	END;
	
\end{lstlisting}

\subsection{Página Renuncia al contrato}
Para realizar la renuncia de un contrato, primero se seleccionan aquellos contratos que tienen una fecha posterior a la actual con una \textbf{acción dinámica}, tal y como en el caso anterior de la \textbf{renovación} a través de una consulta \acrshort{SQL}, guardando el valor de la seg. social en un campo oculto P14\_SS\_OCULTA, ya que es un valor que necesitaremos después, mediante la segunda parte de esta acción en \acrshort{JS}:
\begin{lstlisting}[language=JavaScript, caption={Guarda seguridad social en campo oculto}]
	$s("P14_SS_OCULTA", $v("P14_SEG_SOG"));
\end{lstlisting}

Esta primera parte nos presentará los datos del contrato a renunciar, debiendo introducir después, la fecha de renuncia, que se verificará a través de otra acción dinámica asociada al botón, \textbf{Verificar Fecha}, que comprueba si la fecha de renuncia introducida es menor que la actual fecha fin y si es posterior a la actual, a través del siguiente código \acrshort{JS}:
\begin{lstlisting}[language=JavaScript, caption={Verifica las fecha de renuncia del contrato}]
//  parsear fechas en formato 'DD-MM-YYYY' o 'DD-MON-YYYY'
function parseFechaFlexible(fechaStr) {
	if (!fechaStr) {
		console.warn('Fecha vacia o null:', fechaStr);
		return null;
	}
	
	const partes = fechaStr.trim().split('-');
	if (partes.length !== 3) {
		console.error('Formato inesperado:', fechaStr);
		return null;
	}
	
	const dia = parseInt(partes[0], 10);
	const año = parseInt(partes[2], 10);
	
	// Detectar si el mes es numero o texto
	const mesTexto = partes[1].toUpperCase();
	const meses = {
		'ENE': 0, 'FEB': 1, 'MAR': 2, 'ABR': 3,
		'MAY': 4, 'JUN': 5, 'JUL': 6, 'AGO': 7,
		'SEP': 8, 'OCT': 9, 'NOV': 10, 'DIC': 11
	};
	
	let mes;
	if (!isNaN(parseInt(mesTexto, 10))) {
		mes = parseInt(mesTexto, 10) - 1; // formato numerico
	} else {
		mes = meses[mesTexto]; // formato texto
	}
	
	if (mes === undefined || isNaN(dia) || isNaN(año)) {
		console.error(' Problema al interpretar partes de la fecha:', partes);
		return null;
	}
	
	return new Date(año, mes, dia);
}

// Leer valores 
const fechaInicioStr         = $v('P14_F_INI');
const fechaContratoFinStr    = $v('P14_F_FIN');
const nuevaFechaFinStr       = $v('P14_F_RENUNCIA');
// obtenemos la fecha del actual
const hoy = new Date();
// Extrae año, mes y dia
const añoHoy = hoy.getFullYear();
const mesHoy = hoy.getMonth() + 1; 
const diaHoy = hoy.getDate();
const fechaHoy = new Date(añoHoy, mesHoy - 1, diaHoy); 

// Parsear las fechas
const fechaInicio            = parseFechaFlexible(fechaInicioStr);
const fechaContratoFin       = parseFechaFlexible(fechaContratoFinStr);
const nuevaFechaFin          = parseFechaFlexible(nuevaFechaFinStr);

// Logs para verificar valores
console.log('Fecha INICIO (raw):', fechaInicioStr);
console.log('Fecha FIN actual (raw):', fechaContratoFinStr);
console.log('Nueva fecha FIN (raw):', nuevaFechaFinStr);


console.log('INICIO (obj):', fechaInicio?.toLocaleDateString('es-ES'), typeof fechaInicio);
console.log('FIN actual (obj):', fechaContratoFin?.toLocaleDateString('es-ES'), typeof fechaContratoFin);
console.log('Nueva FIN (obj):', nuevaFechaFin?.toLocaleDateString('es-ES'), typeof nuevaFechaFin);
console.log('fechaHoy (obj):', nuevaFechaFin?.toLocaleDateString('es-ES'), typeof fechaHoy);
// semaforo
let estadoVerificacion = 'ok';
// nueva fecha debe ser posterior a actual
if (nuevaFechaFin && fechaContratoFin) {
	if (nuevaFechaFin > fechaContratoFin) {
		apex.message.alert("AVISO: La fecha de renuncia debe ser anterior a la fecha fin actual");
		$s('P14_F_RENUNCIA', '');
		$('#P14_F_RENUNCIA').trigger('change');
		estadoVerificacion = 'fail';
	}
}
// debe ser posterior a la de inicio y a la actual
if (nuevaFechaFin && fechaContratoFin) {
	if (nuevaFechaFin < fechaInicio || nuevaFechaFin < fechaHoy)  {
		apex.message.alert("AVISO: La fecha de renuncia debe ser posterior a la fecha actual");
		$s('P14_F_RENUNCIA', '');
		$('#P14_F_RENUNCIA').trigger('change');
		estadoVerificacion = 'fail';
	}
}

// cambio color si ok
const \$boton = \$('\#btn\_verifica2');

if (\$boton.length) {
	// Elimina clases anteriores
	\$boton.removeClass('t-Button--success t-Button--hot t-Button--danger');
	
	// Aplica color segun estado
	if (estadoVerificacion === "ok") {
		\$boton.addClass('t-Button--success');
		// HAY QUE CAMBIAR EL FORMATO DE LA SS MOSTRADA
		var ssRaw = \$v("P14\_SEG\_SOG");                    
		var ssClean = ssRaw.replace('E', '')            
		.replace(/s/g, "");     
		
		\$s("P14\_SS\_OCULTA", ssClean);
		
	}
}
\end{lstlisting}

Una vez verificada la fecha, con el botón en verde, y pulsado \textbf{Renunciar}, se ejecuta un proceso \acrshort{PL/SQL}, que reduce las nóminas hasta la nueva fecha fin, y también cambia la de fin del contrato:

\begin{lstlisting}[language=PLSQL, caption={Genera las nuevas nóminas tras la renuncia}]
	DECLARE
	v_fecha_ini        DATE := :P14_F_FIN; 
	v_fecha_fin        DATE := :P14_F_RENUNCIA;
	v_total_meses      PLS_INTEGER;
	v_meses_generados  PLS_INTEGER := 0;
	v_mes_actual       DATE;
	v_ss_mes           NUMBER := :P14_SS_OCULTA;
	v_indemnizacion    NUMBER := :P14_NEW_IND;
	
	v_dni              VARCHAR2(9);
	v_ultimo_mes       DATE;
	
	BEGIN
	EXECUTE IMMEDIATE 'ALTER SESSION SET NLS_NUMERIC_CHARACTERS = '',.''';
	
	-- Obtengo el DNI
	SELECT CONTRATADO
	INTO v_dni
	FROM CONTRATOS
	WHERE REF_CONTRATO = :P14_SEL_CONT;
	DBMS_OUTPUT.PUT_LINE('DNI obtenido: ' || v_dni);
	-- Actualizo la fecha fin del contrato
	UPDATE CONTRATOS
	SET 
	F_FIN = :P14_F_RENUNCIA,
	INDEMNIZACION = v_indemnizacion,
	OBSERVACIONES = 'Renuncia'
	WHERE REF_CONTRATO = :P14_SEL_CONT;
	-- elimino nominas posteriores a la renuncia
	DELETE FROM NOMINA
	WHERE dni_nom = v_dni
	AND TO_DATE(anno || LPAD(mes, 2, '0'), 'YYYYMM') > TRUNC(v_fecha_fin, 'MM');
	
	-- veo cual es ultimo mes de nomina
	SELECT TO_DATE(anno || LPAD(mes, 2, '0'), 'YYYYMM')
	INTO v_ultimo_mes
	FROM (
	SELECT mes, anno
	FROM NOMINA
	WHERE dni_nom = v_dni
	ORDER BY anno DESC, mes DESC
	)
	WHERE ROWNUM = 1;
	
	-- Actualizo ultima numina con su nueva indemnizacion
	UPDATE NOMINA
	SET mensualidad = v_indemnizacion,
	seg_soc = v_ss_mes,
	observaciones = 'Ultimo mes con indemnizacion'
	WHERE dni_nom = v_dni
	AND TO_DATE(anno || LPAD(mes, 2, '0'), 'YYYYMM') = TRUNC(v_fecha_fin, 'MM');
	
	-- Annado el siguiente mes con solo ss
	v_mes_actual := ADD_MONTHS(TRUNC(v_fecha_fin, 'MM'), 1);
	
	INSERT INTO NOMINA (
	dni_nom, mes, anno, mensualidad, seg_soc, observaciones
	) VALUES (
	v_dni,
	TO_NUMBER(TO_CHAR(v_mes_actual, 'MM')),
	TO_NUMBER(TO_CHAR(v_mes_actual, 'YYYY')),
	0,
	v_ss_mes,
	'ultima ss'
	);
	
	END;
\end{lstlisting}

\section{Página de Informes}
En las siguientes páginas se muestran los informes realizados.
\subsection{Página Nómina-mes}
Este informe es el objetivo final de la aplicación, para ello se solicitan primeramente un mes y un año y se realiza una consulta \acrshort{SQL}, al pulsar el botón \textbf{Consultar} y devuelve la suma total del mes para todas las aplicaciones en :
\begin{lstlisting}[language=SQL, caption={Consulta informe nómina para un mes determinado}]
	SELECT 
	o.ORGANICA,
	s.NOM_SOL || ' ' || s.APE1_SOL || ' ' || s.APE2_SOL AS NOMBRE,
	n.MENSUALIDAD,
	n.SEG_SOC,
	n.MENSUALIDAD + n.SEG_SOC AS TOTAL,
	n.OBSERVACIONES,
	:P18_MES AS MES_PEDIDO,
	:P18_ANNO AS ANNO_PEDIDO,
	(
	SELECT SUM(n2.MENSUALIDAD + n2.SEG_SOC)
	FROM PROYECTO o2
	JOIN CONVOCATORIA c2 ON c2.REF_PROY = o2.ORGANICA
	JOIN CONTRATOS ct2 ON ct2.CONVOCATORIA = c2.REFERENCIA
	JOIN SOLICITANTE s2 ON s2.DNI_SOL = ct2.CONTRATADO
	JOIN NOMINA n2 ON n2.DNI_NOM = s2.DNI_SOL
	WHERE n2.MES = :P18_MES
	AND n2.ANNO = :P18_ANNO
	AND (:P18_FILTRO IS NULL OR o2.ORGANICA = :P18_FILTRO)
	) AS TOTAL_GENERAL
	FROM PROYECTO o
	JOIN CONVOCATORIA c ON c.REF_PROY = o.ORGANICA
	JOIN CONTRATOS ct ON ct.CONVOCATORIA = c.REFERENCIA
	JOIN SOLICITANTE s ON s.DNI_SOL = ct.CONTRATADO
	JOIN NOMINA n ON n.DNI_NOM = s.DNI_SOL
	WHERE n.MES = :P18_MES
	AND n.ANNO = :P18_ANNO
	AND (:P18_FILTRO IS NULL OR o.ORGANICA = :P18_FILTRO)
\end{lstlisting}

De este informe se guarda el total en P18\_T\_GEN, a través de una \textbf{acción dinámica} en \acrshort{JS}. Esto lo podría hacer el propio informe dinámico, pero nos interesa para imprimir el informe en pdf:
\begin{lstlisting}[language=JavaScript, caption={Guarda y muestra el total del listado}]
	 $("td[headers='TOTAL_GENERAL']").first().text().trim();
	total = total.replace(",", ".");
	$s("P18_T_GEN", total);
\end{lstlisting}

Mostrado el informe, podemos filtrar la orgánica que queramos y que vuelva a calcular el importe total. Primero mostramos todas las referencias que salen en el informe del mes, con una consulta \acrshort{SQL}:
\begin{lstlisting}[language=SQL, caption={Muestra las referencias de proyectos del informe para seleccionar}]
	SELECT NULL AS display_value, NULL AS return_value
	UNION
	SELECT DISTINCT o.ORGANICA AS display_value, o.ORGANICA AS return_value
	FROM PROYECTO o
	JOIN CONVOCATORIA c ON c.REF_PROY = o.ORGANICA
	JOIN CONTRATOS ct ON ct.CONVOCATORIA = c.REFERENCIA
	JOIN SOLICITANTE s ON s.DNI_SOL = ct.CONTRATADO
	JOIN NOMINA n ON n.DNI_NOM = s.DNI_SOL
	WHERE n.MES = :P18_MES
	AND n.ANNO = :P18_ANNO
	ORDER BY 1
\end{lstlisting}
Y con la acción dinámica anterior se vuelve a refrescar el importe total.

Para poder imprimir el informe a PDF a través de \acrfull{AOP}, es precisa otra acción dinámica que dispare el \textbf{plug-in} \textit{UC-APEX OfficePrint(AOP)-DA}, al pulsar el botón \textbf{Generar PDF}. Indicándole cual es nuestra plantilla personalizada, guardada en shared components: \textbf{plantilla\_informe\_nomina.docx}

\subsection{Página Vencimientos}
En este informe interactivo, se presentarán los vencimientos de contratos entre dos fechas, para su control, a través de una consulta \acrshort{SQL}, al pulsar el botón \textbf{Consulta} que activará la acción dinámica de submit al informe.
\begin{lstlisting}[language=SQL, caption={Muetra de contratatos que vencen en periodo de fechas}]
	SELECT 
	S.APE1_SOL || ' ' || S.APE2_SOL || ', ' || S.NOM_SOL AS CONTRATADO,
	CT.REF_CONTRATO AS CONTRATO, 
	CT.F_FIN AS FIN,
	P.ORGANICA AS ORGANICA
	FROM SOLICITANTE S
	INNER JOIN CONTRATOS CT ON CT.CONTRATADO = S.DNI_SOL
	INNER JOIN CONVOCATORIA CV ON CV.REFERENCIA = CT.CONVOCATORIA
	INNER JOIN PROYECTO P ON P.ORGANICA = CV.REF_PROY
	WHERE CT.F_FIN BETWEEN TO_DATE(:P17_FDESDE, 'DD/MM/YYYY') AND TO_DATE(:P17_FHASTA, 'DD/MM/YYYY')
\end{lstlisting}

\subsection{Página Contratos}
Esta página es un informe interactivo simple que muestra los datos de todos los contratos a través de una consulta \acrshort{SQL}, pudiendo realizar los filtros predeterminados que ofrece \textbf{Actions}:
\begin{lstlisting}[language=SQL, caption={Consulta para todos los contratos}]
	SELECT 
	S.APE1_SOL || ' ' || S.APE2_SOL || ', ' || S.NOM_SOL AS CONTRATADO,
	CT.REF_CONTRATO AS CONTRATO, 
	CT.F_INI AS INICIO, 
	CT.F_FIN AS FIN,
	CV.REFERENCIA AS REF_CONV,  
	P.ORGANICA AS ORGANICA, 
	P.FECHA_INI AS INI_PROY, 
	P.FECHA_FIN AS FIN_PROY
	FROM SOLICITANTE S
	INNER JOIN CONTRATOS CT ON CT.CONTRATADO = S.DNI_SOL
	INNER JOIN CONVOCATORIA CV ON CV.REFERENCIA = CT.CONVOCATORIA
	INNER JOIN PROYECTO P ON P.ORGANICA = CV.REF_PROY
\end{lstlisting}
\subsection{Página Nóminas-Periodo}
Aquí se presentan para un contratado las nóminas que se han pagado en un periodo de tiempo solicitado. Para ello se realiza una primera consulta \acrshort{SQL}, que muestra los contratados:
\begin{lstlisting}[language=SQL, caption={Consulta de contratados para selección}]
	SELECT
	NOM_SOL || ' ' || APE1_SOL || ' ' || APE2_SOL AS display_value,
	DNI_SOL AS return_value
	FROM SOLICITANTE
	JOIN CONTRATOS C ON C.CONTRATADO = DNI_SOL
	ORDER BY NOM_SOL;
\end{lstlisting}
Una vez seleccionado el contratado e introducidas las fechas de inicio y fin se realiza otra consulta \acrshort{SQL}, para rellenar el informe y mostrando el total pagado en ese periodo. En este caso se ha suprimido la opción \textbf{Actions}, ya que el informe es único:
\begin{lstlisting}[language=SQL, caption={Consulta de nóminas para un contratado en un intervalo de tiempo}]
	SELECT 
	n.MES,
	n.ANNO AS ANIO,
	n.MENSUALIDAD AS MENSUALIDAD,
	n.SEG_SOC AS SEG_SOC,
	n.MENSUALIDAD + n.SEG_SOC AS TOTAL,
	n.OBSERVACIONES
	FROM SOLICITANTE s
	JOIN CONTRATOS ct ON s.DNI_SOL = ct.CONTRATADO
	JOIN NOMINA n ON n.DNI_NOM = s.DNI_SOL
	WHERE s.DNI_SOL=:P19_CONTRATADO AND ct.F_INI>= :P19_DESDE AND ct.F_FIN <= :P19_HASTA;
\end{lstlisting}
\apendice{Configuración Técnica de Instalación}

En esta sección se indicarán los parámetros principales para la configuración, tanto de la \acrshort{BDR}, \acrshort{ORDS}, \acrshort{APEX}, \acrshort{AOP} y \textbf{Oracle Cloud}.

\section{Oracel 23ai}
El proceso de instalación se realiza siguiendo las instrucciones de Oracle 23ai ~\cite{DattaInstallingOracleDatabasea}.
Esto instala una \acrfull{PBD}, \textbf{\textbf{FREEPDB1}} donde creamos el esquema de trabajo para nuestro proyecto \textbf{TFGUBU} en "\textit{C:/app/User/product23ai/oradata/FREE/FREEPDB}1"

Una vez instalado y a través de \textbf{SQL plus}, procedemos a crear nuestro entorno conectados como sys as sysdba/1234.
\begin{lstlisting}[language=SQL, caption={Creación de Esquema TFGUBU}]
	-- cambiamos de contenedor
	ALTER SESSION SET CONTAINER = FREEPDB1;
	
	-- creamos nuestro usuario-esquema
	CREATE USER TFGUBU IDENTIFIED BY "tu passwd"
	DEFAULT TABLESPACE USERS
	TEMPORARY TABLESPACE TEMP
	QUOTA UNLIMITED ON USERS;
	
	-- concedemos permisos
	GRANT CONNECT, RESOURCE TO TFGUBU;
	GRANT CREATE SESSION TO TFGUBU;
	GRANT CREATE TABLE, CREATE VIEW, CREATE PROCEDURE TO TFGUBU;
\end{lstlisting}

vemos en la imagen cómo queda configurada la conexión, en nuestro caso para \textbf{FREEPDB1} en el archivo \textbf{tsnames.ora}:
\imagen{tsnamesora}{Configuración conexión a FREEPDB1}
Este archivo permite definir un alias de conexión para acceder al servicio y en la siguiente imagen, vemos como quedaría la configuración del \textbf{listener} que gestiona las solicitudes de conexión entrantes:
\imagenDos{listener}{Configuración Listener a FREEPDB1}{1}

Hay que tener en cuenta que los servicios de Oracle y Listener deben estar activos, sino, no se podrá realizar la conexión.

\imagenDos{services}{Servicios activados de Oracle}{1.2}

Lo que podremos comprobar con el estatus del listener con \textbf{lsnrctl status}:

\imagen{statuslsnr}{Verificación de Listner activo}

A partir de este momento podríamos empezar a crear nuestras tablas con \textbf{SQL Developer} con la conexión indicada, nuestro usuario: \textbf{TFGUBU} y la contraseña establecida.

\section{Oracle REST Data Service}
\acrfull{ORDS} es una herramienta desarrollada por Oracle que permite acceder a sus bases de datos a traves de servicios (RESFULL) \gls{RESTFUL}, esto permite acceder a la BD como si fuera una web sin usar \acrshort{SQL}, si no con peticiones \acrshort{HTTPS}, pudiendo realizar las operaciones \textbf{CRUD}, fácilmente.

Una vez realizada la descarga e instalación (\href{https://www.oracle.com/database/sqldeveloper/technologies/db-actions/download/}{Oracle REST}), lo instalamos localmente, en mi caso: 
\imagenDos{dirords}{Directorio de instalación ORDS}{.7}
En el cual destacamos:
\begin{itemize}
	\item \textbf{ords.war}:Archivo ejecutable principal
	\item \textbf{dir bin}: Scrips de instalación
	\item \textbf{dir databases}:
		\subitem{\textbf{dir default}}: Contiene la configuración por defecto, el subdirectorio \textbf{images} que contiene las imágenes de \acrfull{APEX} y el fichero de configuración de la conexión a la BD \textbf{pool.xml}		
\end{itemize}
Así la conexión entre ORDS y la base de datos se define en este fichero pool.xml donde especificamos el puerto \textbf{1521}.
\imagenDos{pool}{Fichero de configuración pool.xml}{1}

Para el despliegue de la aplicación y una vez que se decidió hacerlo en \acrfull{OCI}, fue necesaria la configuración de \acrshort{HTTPS}. Así se generó un certificado autofirmado privado, con OpenSSL, para ello seguimos los siguientes pasos:
\begin{itemize}
	\item Descargar OpenSSL
	\item Generar la clave privada: \textbf{openssl genrsa -out key.pem 2048}
	\item Crear solicitud de certificado: \textbf{openssl req -new -key key.pem -out cert.csr}
	\item Generar el cert autofirmado: \textbf{openssl x509 -req -in cert.csr -signkey key.pem -out cert.pem -days 365
	}
	\item Crear un archivo combinado: \textbf{\textbf{cat cert.pem key.pem > combinado.pem}} (Requerido por APEX)
\end{itemize}

Una vez hecho esto tendremos que configurar el fichero \textbf{settings.xml}, para que tome este certificado y el puerto 8443.
\imagenDos{settings}{Configuración del fichero settings.xml}{0.9}

Una vez configurado todo podemos acceder a traves del navegador a los servicios \acrshort{ORDS} con \acrshort{HTTPS}:
\imagenDos{webords}{Imagen de acceso web a servicios ORDS}{0.6}

\section{Apex Office Print AOP}

Como uno de los objetivos de la aplicación era crear un informe personalizado desde \acrshort{APEX} en formato \textbf{PDF}, se procedió a instalar la herramienta creada por Oracle para este propósito~\cite{OverviewAPEXOffice}.
Una vez descargada desde \href{https://www.apexofficeprint.com/}{Web AOP}, tendremos que instalar los plug-ins necesarios en nuestra aplicación \textbf{GeNomIn}. Para ello accedemos a \textbf{Shared Components/Plug-ins} y procedemos a importar los ficheros:
\imagen{plugins}{Vista de la instalación de plug-ins de AOP}

Una vez instalados, deberemos proceder a la configuración de cada plug-in para su correcto funcionamiento, en \textbf{Shared Components/Component Settings} escogeremos cada plug-in e introduciremos la web (en este caso al ser local http) y la \textbf{API key}, que nos identifica en el servicio y el modo de AOP, en este caso ya en Producción para que no inserte marca de agua.
\imagen{aopconfig}{Confiuguración de los plug-ins AOP}

Solamente nos quedaría realizar nuestra plantilla personalizada en \textbf{Word}, indicando los campos que deben aparecer:
\imagen{plantillaAOP}{Visualización de la plantilla AOP}
y luego relacionándolo en le informe correspondiente. (\textbf{Informe-nomina})
\imagenDos{accioninforme}{Asociación de la plantilla al informe}{0.8}

\section{Oracle Cloud Infraestructura}
Como ya se comentó en el \textbf{Capítulo 4.5} de la memoria, el mayor problema de \acrshort{OCI}, es el registro, que no lleva asociada más capacidad intelectual que la paciencia.
Una vez conseguido este propósito, procederíamos a crear nuestra instancia de la base de datos como \textbf{Base de datos autónoma}:
\imagen{bdautonoma}{Menu Oracle Database Infraestructure}

En el siguiente paso deberemos configurar los datos principales, teniendo especial cuidado con la región de creación, en nuestro caso \textbf{Spain Central (Madrid)}.
Estableceremos el nombre de la base de datos y la carga de trabajo. Como nosotros vamos a utilizar \acrshort{APEX}, es lo que escogemos.
\imagen{instanciabd}{Creacion de la Instancia de BD-1 en Cloud}
Finalizaremos la configuración escogiendo la opción \textbf{siempre gratis}, seleccionando la versión \textbf{23ai}, como en nuestro modo local, introduciendo la contraseña de Administrador y el modo \textbf{Acceso seguro desde cualquier lugar} para permitir a los usuarios con credenciales, acceso desde Internet.
\imagen{instanciabd2}{Creacion de la Instancia de BD-1 en Cloud}

Una vez creada la base de datos, procederemos a crear la Instancia de \acrshort{APEX}, con similares parámetros:

\imagen{instanciaAPEX}{Creación de Instancia APEX en Cloud}

A partir de este momento podríamos iniciar la instancia de \acrshort{APEX}, importar nuestra base de datos y la aplicación locales y se habría realizado el despliegue. Hay que tener en cuenta que los plug-ins de \acrshort{AOP}, hay que volver a instalar y configurarlos. 
\apendice{Manual de Usuario}
\begin{center}
	\includegraphics[scale=3]{logo.png}
\end{center}

\addcontentsline{toc}{section}{Manual de Usuario}
\label{manual-usuario}

\subsection*{Tabla de contenido del Manual}
\begin{itemize}
	\item \hyperref[objetivo-y-proceso]{Objetivo y proceso} \dotfill \pageref{objetivo-y-proceso}
	\item \hyperref[inicio-de-sesion]{Inicio de Sesión} \dotfill \pageref{inicio-de-sesion} 
	\item \hyperref[administracion]{Administración} \dotfill \pageref{administracion}
	\item \hyperref[responsables]{Responsables} \dotfill \pageref{responsables}
	\item \hyperref[proyectos]{Proyectos} \dotfill \pageref{proyectos}
	\item \hyperref[convocatorias]{Convocatorias} \dotfill \pageref{convocatorias}
	\item \hyperref[solicitante]{Solicitante} \dotfill \pageref{solicitante}
	\item \hyperref[contratos]{Contratos} \dotfill \pageref{contratos}
	\item \hyperref[crear]{Crear contrato} \dotfill \pageref{crear}
	\item \hyperref[renovacion]{Renovación} \dotfill \pageref{renovacion}
	\item \hyperref[renuncia]{Renuncia} \dotfill \pageref{renuncia}
	\item \hyperref[informes]{Informes} \dotfill \pageref{informes}
	\item \hyperref[nomina-mes]{Nómina-mes} \dotfill \pageref{nomina-mes}
	\item \hyperref[vencimientos]{Vencimientos} \dotfill \pageref{vencimientos}
	\item \hyperref[contratos-1]{Inf. Contratos} \dotfill \pageref{contratos-1}
	\item \hyperref[nominas-periodo]{Nóminas-Periodo} \dotfill \pageref{nominas-periodo}
	\item \hyperref[acerca-de]{Acerca de} \dotfill \pageref{acerca-de}
	\item \hyperref[guia-de-uso]{Guía de uso} \dotfill \pageref{guia-de-uso}
\end{itemize}

\section{Objetivo y proceso}\label{objetivo-y-proceso}

El objetivo de la aplicación \textbf{GeNomIn}, es la creación, mantenimiento y gestión de las nóminas vinculadas a proyectos de investigación generados por el Servicio de Investigación de la Universidad de Burgos.

Para ello se ha establecido un proceso por el cual se irán desarrollando
los distintos elementos que conformar un contrato, desde el
mantenimiento de \textbf{Proyectos} e \textbf{Responsables}
(Investigadores), \textbf{Convocatorias}, \textbf{Solicitantes} de las
mismas, creación de \textbf{Contratos}, renovación y renuncia, y el
objetivo final que es la generación de \textbf{Informes} de nómina
mensuales, vencimientos de contratos y controles de nómina, para el
cotejo con los informes emitidos por el Servicio de Retribuciones que
actualmente se llevan con tablas Excel.

En esta guía de uso repasamos el proceso completo desde la
administración de proyectos y responsables hasta el informe final de
nómina mensual.

\section{Inicio de Sesión:}\label{inicio-de-sesion}
	
Una vez accedida a la aplicación nos pedirá las credenciales de entrada, en este caso accederemos con user01, user01.
\imagenDos{1}{Pantalla de Inicio sesión }{.5}

Entrando al Menú principal, encontramos las siguientes opciones en el
Menú Lateral, \textbf{Aministración}, \textbf{Convocatorias},
\textbf{Contratos}, \textbf{Informes} y \textbf{Acerca de}, que pasamos a detallar.

También podremos volver al log inicial pulsando en el botón \textbf{Salir}, de la derecha de la pantalla.

\imagenDos{2}{Pantalla Menú principal}{.6}

\section{Administración}\label{administracion}

Dentro del menú de \textbf{Administración}, podremos realizar la gestión
de los \textbf{Proyectos} y de los \textbf{Responsables} de los mismos

\subsection{Responsables}\label{responsables}

En esta tabla podemos ver y gestionar los Responsables de los proyectos,
indicar que todos estos datos, han sido generados aleatoriamente y no corresponden a la realidad. Como vemos en la imagen, podremos Buscar por campos, (pulsando en el icono de la lupa ) , editar los datos (pulsando en el icono del lápiz ), crear, agrupaciones, filtros, informes y descargarlos, pulsando en el botón \textbf{Actions}.

\imagenDos{3}{Responsables}{.6}

Si lo que queremos es crear un nuevo responsable pulsaremos en la opción \textbf{Create} de la derecha, que nos mostrará el siguiente menú, donde podremos incluir los datos necesarios, DNI, Apellidos, Nombre y
Departamento (el cual podremos escoger en la cortinilla desplegable)

\imagenDos{4}{Pantalla Crear nuevo responsable}{.6}

\subsection{Proyectos}\label{proyectos}

Para los proyectos se ha creado un Grid Interactivo, que muestra los proyectos existentes (éstos también han sido generados de forma aleatoria), realizar búsquedas por Orgánica, Titulo y Responsable, editar campos, añadir nuevos proyectos (\textbf{Add Row}) y desde el botón \textbf{Actions}, generar filtros e informes.

\imagen{5}{Pantalla de Proyectos}

\section{Convocatorias}\label{convocatorias}

\imagen{6}{Convocatorias}
Desde el menú convocatorias, podremos realizar el \textbf{Mantenimiento} de las mismas que solicite un Responsable de un Proyecto. Inicialmente nos mostrará todas las que tenemos, pudiendo editarlas pulsando el icono del lapicero

También podremos hacer búsquedas por los distintos campos, pulsando como en anteriores ocasiones en la lupa y las mismas acciones.

Para generar una nueva convocatoria pulsaremos en el botón \textbf{Create}:

Tendremos que rellenar los campos que se indican obligatorios y marcados con un \textbf{*}

En el desplegable \textbf{Tit Requerida}, podremos seleccionar una de las titulaciones mínima que se requerirán para poder optar, por los solicitantes a cada convocatoria.

En el desplegable \textbf{Investigador}, podremos seleccionar cualquiera de los investigadores que estén dados de alta en el sistema, como hemos visto anteriormente en la pantalla \textbf{Responsables}. Este punto es importante, ya que en el siguiente desplegable \textbf{Ref. Proy}, solo nos mostrará los proyectos del investigador seleccionado, indicándonos después el tiempo que queda para la finalización de este proyecto.

Al confirmar con \textbf{OK} podremos pulsar en el botón
\textbf{Create}, generando una nueva convocatoria.

\imagenDos{8}{Pantalla de Nueva convocatoria}{.6}

\subsection{Solicitante}\label{solicitante}

Esta pantalla nos mostrará los solicitantes existentes para las convocatorias creadas, pudiendo realizar las mismas operaciones que anteriormente (búsquedas por campos y diversas acciones)

También podremos dar de alta un nuevo solicitante de una convocatoria abierta, pulsando en el botón \textbf{Create}.

\imagenDos{9}{Pantalla de Nuevo solicitante}{.5}

Tendremos que rellenar los campos que se indican y seleccionar en el desplegable \textbf{Titulación}, una de las opciones que se indican correspondiente al nivel académico del solicitante.

Luego escogeremos la convocatoria en la que quiere participar, y al seleccionarla, el sistema nos informará si puede o no participar, según el nivel requerido para la misma. En este caso con un nivel de Bachiller, en la convocatoria para Test, nos avisa de la incidencia.

\imagenDos{10}{Aviso nivel académico insuficiente}{.6}

También, nos indicará si la convocatoria ya está cerrada, es el caso de si escogemos la opción de convocatoria cerrada.

Si la convocatoria fuera adecuada a su titulación pulsaríamos el botón \textbf{Create}

\section{Contratos}\label{contratos}

\subsection{Crear}\label{crear}

Dentro de esta opción podremos \textbf{Crear} un contrato para un \textbf{Solicitante}, vinculado a una \textbf{Convocatoria}

\imagen{13}{Pantalla de Contratos}

Inicialmente vemos los contratos dados de alta, pudiendo realizar las distintas tareas como en anteriores ocasiones, tales como, búsquedas por campos y diversos filtros e informes. Además, pulsando en el icono de edición podremos modificar datos de un contrato.

Para generar un nuevo contrato, pulsaremos en el botón \textbf{Create}

Inicialmente nos solicitará para qué convocatoria queremos el contrato, mostrando aquellos solicitantes de esa convocatoria.

\imagenDos{14}{Pantalla selección contratado}{.5}

Luego nos irá solicitando los diversos datos para realizar el contrato.

En el desplegable \textbf{Tipo}, podremos escoger entre los distintos tipos de contrato en vigor.

Después deberemos introducir las fechas de inicio y fin, las cuales deberán estar comprendidas entre las fechas del \textbf{Proyecto} vinculado, lanzando sendos avisos si no se cumpliera este requisito:

\imagenDos{15}{Aviso de fechas incorrectas en contrato}{.6}

Una vez introducidos los datos requeridos y pulsando en el botón \textbf{Create}, nos pedirá confirmación para la creación del nuevo contrato y de la generación las pagos en la tabla nómina automáticamente, para los meses correspondientes y uno adicional de seguridad social, que no tiene retribución.

\imagenDos{16}{Vista de la tabla de nómina}{.8}

\subsection{Renovación}\label{renovacion}

En este apartado podremos realizar la renovación de un contrato, siempre y cuando no sobrepase la fecha fin del \textbf{Proyecto}  vinculado.

Así, al seleccionar un contrato, nos ofrecerá los datos del mismo en la parte izquierda:

\imagenDos{17}{Pantalla de Renovación de contrato}{.9}

Pudiendo introducir los nuevos datos en la parte derecha.

Para la fecha de renovación deberemos verificar la misma, pulsando en el botón \textbf{Verifica Fecha} indicándonos si por ejemplo, la fecha es anterior a la actual, si la fecha excede del fin del contrato o si la fecha fuese correcta, el botón cambiará a color verde.
\imagen{19}{Aviso fecha contrato excede proyecto}
Una vez rellenos los datos correspondientes y pulsado el botón \textbf{Renovar}, nos solicitará confirmación, como en el caso de creación inicial, generará las nuevas nóminas correspondientes y cambiará la fecha fin del mismo.

\imagenDos{20}{Aviso confirmación renovación contrato}{.8}

Realizando la confirmación:

\imagenDos{21}{Aviso prórroga realizada}{.6}

\subsection{Renuncia}\label{renuncia}

En esta sección un solicitante podrá renunciar a un contrato en una fecha determinada. Como en el caso de la \textbf{Renovación}, tendremos que  seleccionar un contrato:

\imagen{22}{Pantalla de renuncia de contrato}

El sistema nos mostrará los datos del contrato y la fecha fin actual, debiendo introducir en la parte de la derecha los datos de renuncia, y como en el caso anterior, comprobará si la renuncia es posterior a la fecha actual.

\imagenDos{23}{Aviso renuncia anterior a fecha actual}{.5}

Y si es anterior a la fecha fin actual.

\imagenDos{24}{Aviso renuncia posterior a fin contrato}{.5}

En el caso de haber introducido una fecha de renovación correcta, se verifica con el botón en verde y al completar el resto de datos podremos pulsar en el botón \textbf{Renunciar}

\imagenDos{25}{Botón renunciar contrato}{.5}

En este momento se nos pedirá confirmación para realizar la acción, que eliminará las nóminas correspondientes y cambiará la fecha fin del contrato.

\imagenDos{26}{Confirmación renuncia contrato}{.5}

\section{Informes}\label{informes}

\subsection{Nómina-mes}\label{nomina-mes}

Con este informe podremos ver la nómina correspondiente al mes seleccionado. Es el informe principal de la aplicación, ya que se compara con la sábana de retribuciones y coteja con los datos obrantes en el servicio.

Para ello accedemos a la sección de \textbf{informes Nómina-mes} que muestra la pantalla siguiente y en la que podemos escoger mes y año, pulsar en Consultar y obtendremos las nóminas a pagar ese mes.

\imagen{28}{Página Informe nómina-mes}

Una vez mostrados los datos, podremos filtrar por orgánica y volver a pulsar Consultar para aplicar ese filtro.
Conforme la vista, pulsando en el botón generar, nos descargará un informe personalizado en pdf a través del servicio online \acrfull{AOP}.

\imagen{30}{Informe personalizado con AOP}

\subsection{Vencimientos}\label{vencimientos}

Desde aquí podremos consultar los vencimientos de los contratos. Para ello, introduciremos la fecha inicio y fin y pulsaremos el botón Consulta, que nos mostrará los contratos que vencen entre las fechas indicadas:

\imagen{32}{Página de Informes por Vencimientos}

Pulsando en la lupa podremos filtrar por campos y el botón \textbf{Actions}, nos permite realizar varias opciones; como columnas a mostrar, filtros avanzados, ordenar datos y operar, guardar este informe para otras ocasiones (si lo hemos modificado) y descargar el informe en varios formatos.

\subsection{Contratos}\label{contratos-1}

Este informe ofrece un listado de todos los datos de los contratos que hay en vigor.

\imagen{33}{Página de Informes de todos los contratos}

Como en el anterior informe se pueden filtrar campos con la lupa y acceder a otras opciones y filtros avanzados con el botón \textbf{Actions}.

\subsection{Nominas-Periodo}\label{nominas-periodo}

Este informe presenta las nóminas que tiene un contratado en un periodo que se determinado. Una vez accedido al informe tendremos que seleccionar el contratado y el periodo desde - hasta.

\imagen{34}{Página de Informe de nóminas por periodo}
Una vez pulsado el botón Consulta, se nos muestran las nóminas y el total de ese periodo. Este informe es usado para cotejar el importe pagado en un periodo a un contratado.
\imagenDos{35}{Informe de nóminas por periodo}{1}

\section{Acerca de}\label{acerca-de}

Se muestran datos de la aplicación:\\
\imagenDos{36}{Página de Información de la app}{.6}
\subsection{Guía de uso}\label{guia-de-uso}

\imagen{37}{Página de la guía de uso}

\apendice{Competencias de Sostenibilidad Curricular}

Como ya se expuso brevemente en el capítulo 3.8 de la memoria, durante el desarrollo de este trabajo y la aplicación \textbf{GeNomIn}, que ha sido creada para modernizar un proceso administrativo tedioso. Desarrollada en \acrshort{APEX} y desplegada en \acrshort{OCI}, se ha intentado integrar los \acrfull{ODS} en el ámbito tecnológico ~\cite{MarkiegiIntegrandoODSGrado}.

Este proyecto me ha permitido comprender y aplicar estas competencias en sostenibilidad curricular, que se alinean con el \textbf{uso responsable de recursos}, la\textbf{ conciencia ambiental}, la \textbf{participación comunitaria} y los \textbf{principios éticos}.

\section{Competencias Adquiridas}
Este \acrshort{TFG}, ha permitido reflexionar sobre la transformación que está sufriendo la sociedad, en particular las administraciones, influidas por la tecnología. En particular vemos como la digitalización reduce el impacto ambiental, mejora el acceso a la información y es mucho más eficiente que el mero uso del papel.

\subsection{Uso sostenible de recursos}

Según los datos aportados por (\href{https://www.oracle.com/es/sustainability/}{Oracle y sostenibilidad}) sus centros utilizan energía renovable y diseño circular. Además, la eliminación de procesos manuales y en papel, supone una reducción de un 90\% de éstos, haciendo el proceso más sostenible en un entorno en la nube.

\subsubsection{Participación comunitaria}
En el desarrollo del proyecto, como no podía ser de otra forma, se involucró al personal del Servicio para un mejor control del proceso administrativo seguido en el proyecto. Esto promueve una transformación digital transparente dentro de las instituciones.

\subsubsection{Pricipios éticos}
El uso, por parte de Oracle, de herramientas \acrfull{ESG}, refuerza el compromiso ético con la sostenibilidad, alienado con los abjetivos \acrshort{ODS}, \textbf{12: (producción
y consumo responsables)}, \textbf{13: Acción por
el clima} y \textbf{16: Paz, justicia e instituciones sólidas}

\section{Aplicación en el Proyecto GeNomIn}
\subsection{Diseño y Funcionalidad}
\textbf{GeNomIn} ha sido diseñada para sustituir hojas de cálculo en almacenamiento compartido por una solución segura, escalable e integral. Esta automatización puede reducir los errores hasta en un 70\% y un ahorro temporal de más del 30\%
\subsection{Impacto Ambiental y Social}
La digitalización de procesos administrativos es uno de los objetivos \acrshort{ODS} (9), que complementa al objetivo (13), \textbf{acción climática}, y a las \textbf{instituciones sólidas} (16). Así, esta migración a la nube supone, con servicios optimizados con gran eficiencia energética, un impacto ambiental mínimo.
\subsection{Conciencia ambiental}
Durante la realización del proyecto se reflexionó sobre la ideoneidad de la migración a un entorno en la nube como Oracle y el consumo de sus servidores. Pero, evidentemente, el impacto ambiental es mucho mayor en el entorno de la propia Universidad que no dispone de los recursos de Oracle, aunque tampoco se ha podido valorar esta diferencia.

\section{Conclusión}

Lo cierto es que antes de la iniciación de este \acrshort{TFG}, no se había reflexionado sobre el impacto que puede tener la tecnología en el medio ambiente. Todos oímos hablar del coste (economico-ambiental) de los grandes servidores de Google, pero no caemos en la cuenta de nuestro propio trabajo.
Analizando el uso anterior, a través de hojas de cálculo almacenamientos, impresiones en papel y procesos obsoletos, nos damos cuenta de la importancia de adquirir estas competencias, no solo en el ámbito de la informática, si no, en la vida cotidiana.

\textbf{GeNomIn} es solo una gota de agua en la transformación de la administración hacia procesos más modernos y eficientes, pero me ha servido de aprendizaje para comprender el impacto que tienen las decisiones técnicas que se alinean con los \acrshort{ODS} y la agenda 2030.

\printnoidxglossary[type=\acronymtype]
\printnoidxglossary[]

\bibliographystyle{plain}
\bibliography{bibliografia}

\end{document}
